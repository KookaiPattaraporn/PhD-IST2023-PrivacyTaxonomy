\documentclass{article}
\usepackage[a4paper, total={6in, 8in}]{geometry}

\geometry{
	a4paper,
	total={170mm,257mm},
	left=20mm,
	top=20mm,
}

\usepackage[utf8]{inputenc}
\usepackage{booktabs} % For formal tables
%\usepackage{url}
\usepackage{lscape}
\usepackage{hyperref}
\usepackage{color}
\usepackage{soul}
%\usepackage[nocompress]{cite}
\usepackage{booktabs}
\usepackage{longtable}
\usepackage{graphicx}
\usepackage{multirow}
\usepackage{multicol}
\usepackage{mathrsfs}
\usepackage{amssymb}
\usepackage{amsbsy}
\usepackage{amsmath}
\usepackage{tcolorbox}
\usepackage{balance}
\usepackage{microtype}
\usepackage{array}
\usepackage{longtable}
\usepackage{indentfirst}
\usepackage{pgfplots}
\usepackage{enumitem}
\usepackage{algorithmic}
\usepackage{textcomp}
\usepackage{xcolor}
\usepackage{eurosym} %\EUR{arg1}
\usepackage[numbers]{natbib}

\begin{document}
\textbf{\Large{Supplementary Materials for Mining and Classifying Privacy and Data Protection Requirements in Issue Reports Manuscript}}

\section{Additional examples for privacy requirements identification (Section 4.1)}

This section provides more examples of the privacy requirements derived in the privacy requirements identification process.

It is important to note that we have followed the GBRAM to extract and refine requirements from narrative statements. For some statements, they are straightforward since privacy requirements can be directly derived from them. However, we have restructured and refined some privacy requirements to emphasise the functionalities that should be provided by software systems. For example, Art. 7(1) in the GDPR states ``Where processing is based on consent, the controller shall be able to demonstrate that the data subject has consented to processing his or her personal data''. From this statement, we derive requirement \emph{R17 \textbf{SHOW} the relevant stakeholders the consent given by the data subjects to process their personal data}.

The lawfulness of processing is one of the key principles for protecting privacy of data subjects. Based on the GDPR Art. 6, the processing of personal data will be lawful if the data subjects give consent \emph{or} the processing is required for other legal conditions, such as the processing is necessary for the performance of contract, compliance with a legal obligation and protecting vital interests. R35 in the taxonomy addresses the requirement of obtaining consent from the data subjects for the processing. For other legal conditions, the data controllers must inform the data subjects if they need to process personal data under those conditions. Requirements R38 and R39 in the taxonomy cover these scenarios.

After lawfully obtaining personal data, the data controllers must provide the data subjects with mechanisms to execute their individual rights in the system. All the privacy requirements related to the rights of data subjects are covered in our taxonomy (i.e., the right to be informed (R12, R24, R26, R30, R31 and R50), the right of access (R1), the right to rectification (R45), the right to erasure (R44), the right to restriction of processing (R4), the right to data portability (R33), the right to object (R3), and the rights in relation to automated decision making and profiling (R21)).

A number of requirements (e.g. R35, R39 and R60) can be triggered when the software is dealing with special category data/more sensitive data. In addition to consent, there are other conditions for a lawful processing of special categories of personal data (e.g. necessary for protecting vital interests and public interests). Requirement R35 covers the consent condition. R39 and R60 in our taxonomy address the remaining conditions specified in GDPR Art. 9. R39 requires the data controllers, if not obtaining the consent, must provide the data subjects the purpose(s) of the processing of special categories of personal data to the data subjects. The controllers also require to protect those personal data with appropriate measures (R60). The key privacy requirements related to international data transfers are also covered as follows: i) the data subjects must be informed about the transfer of their personal data to a third country or an international organisation (R9); ii) the personal data must be appropriately protected (R60) and iii) the transfer of personal data must comply with local requirements (R65).

We note that an article, a section or a statement can lead to the identification of more than one requirements. For example, we derived 17 privacy requirements (i.e. R1-R4, R6, R9, R20-22, R27, R29, R34, R37, R39, R44-R45 and R55) from the Art. 13 in GDPR\footnote{See the file \emph{Privacy-requirements-references} in the replication package for more details \cite{reppkg-pridp}.}. The following example demonstrates two privacy requirements that were derived from a statement. A statement in Section 23-6 in Thailand PDPA states ``In collecting the Personal Data, the Data Controller shall inform the data subject, ... (5) information, address and the contact channel details of the Data Controller, where applicable, of the Data Controller's representative or data protection officer;'', we derived two requirements from this statement which are \textit{R20 PROVIDE the data subjects with the contact details of a data protection officer (DPO)} and \textit{R22 PROVIDE the data subjects with the identity and contact details of a controller/controller's representative}.

We also note that we excluded the articles related to the DPOs in our study due to the following reasons. Firstly, the articles related to DPOs in GDPR mainly focus on their duties, tasks and responsibilities (i.e. Art. 37 - 39). The DPOs requirements are related to governance aspect rather than software requirements aspect. Secondly, the main role that directly determines the activities related to the processing of personal data is the data controllers. This makes the data controllers the key stakeholder in governing how personal data is processed and how the processing activities should be done in software development level. Finally, we have not found any DPOs requirements reported in issue reports in our study. This finding implies that the DPOs requirements were not reflected as software requirements in this context.

\newpage
\section{Comparing the level of abstraction between the privacy requirements derived from GDPR, ISO/IEC 29100, Thailand PDPA and APEC privacy framework (Section 4.2)}

The privacy requirements derived from GDPR, ISO/IEC 29100, Thailand PDPA and APEC privacy framework are in fact at the same level of abstraction. Table \ref{tab:GDPR-ISO-req-mapping} below presents the examples demonstrating the privacy requirements that were derived from different regulations and frameworks, but refer to similar things. For example, we derived \textit{\textbf{PROVIDE} the data subject the recipients or categories of recipients of the personal data} from Art. 13(1)(e) in GDPR, \textit{\textbf{PROVIDE} the types of persons whom the PII can be transferred} from openness, transparency and notice principle in ISO/IEC 29100, \textit{\textbf{INFORM} the data subject the categories of Persons or entities to whom the collected Personal Data may be disclosed} from Section 23 in Thailand PDPA and \textit{\textbf{PROVIDE} the types of persons or organisations to whom personal information might be disclosed} from Point 21 in APEC framework. These four requirements demonstrate that they are at the same level of abstraction and aim to achieve the same goal. They can be merged in the requirements refinement process. \citeauthor{Meis} has also confirmed that GDPR and ISO/IEC 29100 are at the same level of abstraction \cite{Meis}. 

\begin{landscape}
	\begin{table}
		\centering
		\caption{Some sample requirements derived from GDPR, ISO/IEC 29100, Thailand PDPA and APEC privacy framework demonstrate that they are at the same level of abstraction.}
		\label{tab:GDPR-ISO-req-mapping}
		\begin{tabular}{p{4cm} p{1.25cm} p{4cm} p{1.25cm} p{4cm} p{1.25cm} p{4cm} p{1.25cm}}
			\toprule
			\textbf{Requirements derived from GDPR statements} & \textbf{GDPR reference} & \textbf{Requirements derived from ISO/IEC 29100} & \textbf{ISO/IEC 29100 reference} &
			\textbf{Requirements derived from Thailand PDPA} & \textbf{Thailand PDPA reference} &
			\textbf{Requirements derived from APEC framework} & \textbf{APEC framework reference} \\
			\midrule
			PROVIDE the existence of the right to withdraw consent & 13(2)(c)
			& ALLOW a PII principal to withdraw consent & 5.2
			& ALLOW the data subject to withdraw his or her consent & 19-5
			& None & None \\
			PROVIDE the data subject the recipients or categories of recipients of the personal data & 13(1)(e)
			& PROVIDE the types of persons whom the PII can be transferred & 5.8
			& INFORM the data subject the categories of Persons or entities to whom the collected Personal Data may be disclosed & 23-5
			& PROVIDE the types of persons or organisations to whom personal information might be disclosed & 21-4\\
			PROVIDE the data subject the purposes of the processing & 13(1)(c)
			& PROVIDE the PII principal the purpose of the processing of PII & 5.8
			& INFORM the data subject the purpose of the collection, use, or disclosure of the Personal Data & 19-3
			& PROVIDE the individuals with the purpose their information is to be used & (21-23)-1 \\
			PROVIDE the data subject the categories of personal data concerned & 15(1)(b)
			& PROVIDE the PII principal the specified PII required for the specified purpose & 5.8
			& INFORM the data subject the Personal Data to be collected & 23-4
			& PROVIDE the individuals with what personal information is collected & (21-23)-1 \\
			IMPLEMENT appropriate technical and organisational measures to protect personal data & 32(1)(a)
			& PROTECT PII with appropriate controls & 5.11
			& PROVIDE appropriate security measures for preventing the Personal Data & 37-2
			& IMPLEMENT organizational controls to prevent from the wrongful collection or misuse of personal information & 20-1, 28-1, 28-2\\ \bottomrule
		\end{tabular}
	\end{table}	
\end{landscape}

\begin{comment}
\newpage
\section{Traceability between issue reports and privacy concerns expressed in the regulations, standards and frameworks} %Section 4.3

Our work helps establish the traceability between issue reports and privacy concerns expressed in the regulations, standards and frameworks. These suggest what should be done to resolve the concerns raised in issue reports and meet the associated privacy requirements. In addition, this traceability facilitates compliance checking with respect to some specific regulations, standards and/or frameworks. The following examples demonstrate the privacy requirements traceability in issue reports. Referring to issue 123403 in Section 4.2, this bug-regression issue report concerns requirement R44 in the taxonomy. Requirement R44 was derived from GDPR, ISO/IEC 29100, Thailand PDPA and APEC privacy framework. To resolve this issue and comply with all four regulations, standards and frameworks, the system must provide a functionality for the controller to allow the data subjects to erase their personal data. In another example, referring to issue 123403 in Section 4.2, it addresses two privacy requirements, R30 and R44. Requirement R30 was derived from GDPR and APEC privacy framework. To resolve this issue and to comply with GDPR and APEC privacy framework, the system must also provide a functionality for the controller to provide the data subjects the information relating to the processing of personal data with standardised icons.

Our approach can be also applied to establish privacy requirements traceability in other software artifacts such as design models, source code and test cases. In general, our approach can be used a reference framework in developing privacy-aware software systems. For example, when a software engineer develops a system functionality which collects personal data, the software engineer may consult the taxonomy to identify the requirements related to the information required to provide to the data subjects before collecting their personal data such as concerned personal data (R42), purposes of collection (R38), purposes of processing (R39) and period/criteria used to store personal data (R55). The privacy requirements were derived from the regulations, standards and frameworks, hence meeting those requirements forms a basis for privacy compliance.

Although issue reports are a good source of information for software requirements, we acknowledge that they are not the only source. Software requirements can be in other forms such as requirements specifications or other documents (such as Confluence\footnote{https://www.atlassian.com/software/confluence/features} pages). The privacy requirements in the taxonomy can also be mapped to a list of privacy-related items concerned in a range of checks in organisations such as Data Protection Impact Assessment (DPIA). For example, DPIA requires a system to obtain consent from data subjects before processing their personal data. Requirement R35 in the taxonomy addresses this concern. However, the scope of this work focuses on privacy requirements in issue reports, and we plan to explore privacy requirements in other sources in our future work.
\end{comment}

\begin{comment}
\newpage
\section{Number of issue reports in Chrome and Moodle projects counted by issue type (Section 7.1)}

Table \ref{tab:issue-type} shows number of issue reports in Google Chrome and Moodle datasets categorised by issue type.

\begin{table}[h]
	\centering
	\caption{Number of issue reports in Chrome and Moodle projects counted by issue type.}
	\label{tab:issue-type}
	\begin{tabular}{p{3cm} p{2cm} p{3cm} p{2cm}}
		\toprule
		\textbf{Chrome} & \textbf{\#issues} & \textbf{Moodle} & \textbf{\#issues} \\
		\midrule
		Bug & 620 & Bug & 223 \\
		Bug-regression & 59 & Epic & 3 \\
		Bug-security & 36 & Improvement & 101 \\
		Feature & 132 & New Feature & 75 \\
		Task & 5 & Task & 26 \\
		Unspecified & 44 & Sub-task & 37 \\
		~ & ~ & Functional test & 13 \\
		\bottomrule
		\textbf{Total} & \textbf{896} & \textbf{Total} & \textbf{478} \\
		\bottomrule
	\end{tabular}
\end{table}


\newpage
\section{Privacy requirements occurrences (Section 7.1)}

Figure \ref{fig:top10} shows top 10 privacy requirements occurrences in Google Chrome and Moodle datasets categorised by issue type.

\begin{figure}[ht]
	\centering
	\includegraphics[width=0.8\linewidth]{Figures/"Top10-occ-by-type"}
	\caption{Top 10 privacy requirements occurrences in Google Chrome and Moodle datasets categorised by issue type}
	\label{fig:top10}
\end{figure}
\end{comment}

\newpage
\section{Top 10 privacy requirements and their descriptive statistics (Section 7.1)}

The following tables show the top 10 privacy requirements mined in Chrome and Moodle projects together with their frequency based on issue type.

\begin{table}[ht]
\centering
\caption{A table summarising top 10 privacy requirements with their frequency based on issue type in Chrome project.}
\label{tab:top10-stats-chrome}
\resizebox{7in}{!}{%
\begin{tabular}{|l|l|l|l|l|l|l|}
\hline
\textbf{Project} & \textbf{Requirement} & \textbf{Category} & \textbf{Subcategory} & \textbf{Issue type} & \textbf{Frequency} & \textbf{Total frequency} \\ \hline
Chrome & R30 & 2) Notice & 2.1) Data subjects & Bug & 151 & 209 \\ 
~ & ~ & ~ & ~ & Bug-regression & 22 & ~ \\
~ & ~ & ~ & ~ & Bug-security & 1 & ~ \\
~ & ~ & ~ & ~ & Feature & 32 & ~ \\
~ & ~ & ~ & ~ & Task & 1 & ~ \\
~ & ~ & ~ & ~ & Unspecified & 2 & ~ \\ \hline
Chrome & R44 & 1) User participation & - & Bug & 151 & 204 \\
~ & ~ & ~ & ~ & Bug-regression & 15 & ~ \\
~ & ~ & ~ & ~ & Bug-security & 4 & ~ \\
~ & ~ & ~ & ~ & Feature & 21 & ~ \\
~ & ~ & ~ & ~ & Task & 1 & ~ \\
~ & ~ & ~ & ~ & Unspecified & 12 & ~ \\ \hline
Chrome & R60 & 7) Security & - & Bug & 80 & 135 \\
~ & ~ & ~ & ~ & Bug-regression & 4 & ~ \\
~ & ~ & ~ & ~ & Bug-security & 25 & ~ \\
~ & ~ & ~ & ~ & Feature & 17 & ~ \\
~ & ~ & ~ & ~ & Task & 1 & ~ \\
~ & ~ & ~ & ~ & Unspecified & 8 & ~ \\ \hline
Chrome & R8 & 3) User desirability & 3.1) Consent & Bug & 89 & 144 \\
~ & ~ & ~ & 3.3) Preference & Bug-regression & 4 & ~ \\
~ & ~ & ~ & ~ & Bug-security & 25 & ~ \\
~ & ~ & ~ & ~ & Feature & 17 & ~ \\
~ & ~ & ~ & ~ & Task & 1 & ~ \\
~ & ~ & ~ & ~ & Unspecified & 8 & ~ \\ \hline
Chrome & R36 & 3) User desirability & 3.2) Choice & Bug & 74 & 119 \\
~ & ~ & ~ & ~ & Bug-regression & 7 & ~ \\
~ & ~ & ~ & ~ & Bug-security & 4 & ~ \\
~ & ~ & ~ & ~ & Feature & 32 & ~ \\
~ & ~ & ~ & ~ & Task & 0 & ~ \\
~ & ~ & ~ & ~ & Unspecified & 2 & ~ \\ \hline
Chrome & R45 & 1) User participation & - & Bug & 58 & 73 \\
~ & ~ & ~ & ~ & Bug-regression & 1 & ~ \\
~ & ~ & ~ & ~ & Bug-security & 0 & ~ \\
~ & ~ & ~ & ~ & Feature & 11 & ~ \\
~ & ~ & ~ & ~ & Task & 0 & ~ \\
~ & ~ & ~ & ~ & Unspecified & 3 & ~ \\ \hline
Chrome & R53 & 4) Data processing & 4.4) Erasure & Bug & 46 & 70 \\
~ & ~ & ~ & ~ & Bug-regression & 4 & ~ \\
~ & ~ & ~ & ~ & Bug-security & 2 & ~ \\
~ & ~ & ~ & ~ & Feature & 9 & ~ \\
~ & ~ & ~ & ~ & Task & 0 & ~ \\
~ & ~ & ~ & ~ & Unspecified & 9 & ~ \\ \hline
Chrome & R1 & 1) User participation & - & Bug & 31 & 49 \\
~ & ~ & ~ & ~ & Bug-regression & 6 & ~ \\
~ & ~ & ~ & ~ & Bug-security & 0 & ~ \\
~ & ~ & ~ & ~ & Feature & 11 & ~ \\
~ & ~ & ~ & ~ & Task & 0 & ~ \\
~ & ~ & ~ & ~ & Unspecified & 1 & ~ \\ \hline
Chrome & R26 & 2) Notice & 2.1) Data subjects & Bug & 20 & 29 \\
~ & ~ & ~ & ~ & Bug-regression & 1 & ~ \\
~ & ~ & ~ & ~ & Bug-security & 0 & ~ \\
~ & ~ & ~ & ~ & Feature & 6 & ~ \\
~ & ~ & ~ & ~ & Task & 1 & ~ \\
~ & ~ & ~ & ~ & Unspecified & 1 & ~ \\ \hline
Chrome & R41 & 4) Data processing & 4.1) Collection & Bug & 12 & 17 \\
~ & ~ & ~ & ~ & Bug-regression & 1 & ~ \\
~ & ~ & ~ & ~ & Bug-security & 0 & ~ \\
~ & ~ & ~ & ~ & Feature & 3 & ~ \\
~ & ~ & ~ & ~ & Task & 0 & ~ \\
~ & ~ & ~ & ~ & Unspecified & 1 & ~ \\ \hline
\end{tabular}%
}
\end{table}\begin{table}[ht]
	\centering
	\resizebox{7in}{!}{%
	\begin{tabular}{|l|l|l|l|l|l|l|}
		\hline
		\textbf{Project} & \textbf{Requirement} & \textbf{Category} & \textbf{Subcategory} & \textbf{Issue type} & \textbf{Frequency} & \textbf{Total frequency} \\ \hline
		Chrome & R30 & 2) Notice & 2.1) Data subjects & Bug & 151 & 209 \\ 
		~ & ~ & ~ & ~ & Bug-regression & 22 & ~ \\ 
		~ & ~ & ~ & ~ & Bug-security & 1 & ~ \\ 
		~ & ~ & ~ & ~ & Feature & 32 & ~ \\ 
		~ & ~ & ~ & ~ & Task & 1 & ~ \\ 
		~ & ~ & ~ & ~ & Unspecified & 2 & ~ \\ \hline
		Chrome & R44 & 1) User participation & - & Bug & 151 & 204 \\ 
		~ & ~ & ~ & ~ & Bug-regression & 15 & ~ \\ 
		~ & ~ & ~ & ~ & Bug-security & 4 & ~ \\ 
		~ & ~ & ~ & ~ & Feature & 21 & ~ \\ 
		~ & ~ & ~ & ~ & Task & 1 & ~ \\ 
		~ & ~ & ~ & ~ & Unspecified & 12 & ~ \\ \hline
		Chrome & R60 & 7) Security & - & Bug & 80 & 135 \\ 
		~ & ~ & ~ & ~ & Bug-regression & 4 & ~ \\ 
		~ & ~ & ~ & ~ & Bug-security & 25 & ~ \\ 
		~ & ~ & ~ & ~ & Feature & 17 & ~ \\ 
		~ & ~ & ~ & ~ & Task & 1 & ~ \\ 
		~ & ~ & ~ & ~ & Unspecified & 8 & ~ \\ \hline
		Chrome & R8 & 3) User desirability & 3.1) Consent & Bug & 89 & 129 \\ 
		~ & ~ & ~ & 3.3) Preference & Bug-regression & 4 & ~ \\ 
		~ & ~ & ~ & ~ & Bug-security & 25 & ~ \\ 
		~ & ~ & ~ & ~ & Feature & 17 & ~ \\ 
		~ & ~ & ~ & ~ & Task & 1 & ~ \\ 
		~ & ~ & ~ & ~ & Unspecified & 8 & ~ \\ \hline
		Chrome & R36 & 3) User desirability & 3.2) Choice & Bug & 74 & 119 \\ 
		~ & ~ & ~ & ~ & Bug-regression & 7 & ~ \\ 
		~ & ~ & ~ & ~ & Bug-security & 4 & ~ \\ 
		~ & ~ & ~ & ~ & Feature & 32 & ~ \\ 
		~ & ~ & ~ & ~ & Task & 0 & ~ \\ 
		~ & ~ & ~ & ~ & Unspecified & 2 & ~ \\ \hline
		Chrome & R45 & 1) User participation & - & Bug & 58 & 73 \\ 
		~ & ~ & ~ & ~ & Bug-regression & 1 & ~ \\ 
		~ & ~ & ~ & ~ & Bug-security & 0 & ~ \\ 
		~ & ~ & ~ & ~ & Feature & 11 & ~ \\ 
		~ & ~ & ~ & ~ & Task & 0 & ~ \\ 
		~ & ~ & ~ & ~ & Unspecified & 3 & ~ \\ \hline
		Chrome & R53 & 4) Data processing & 4.4) Erasure & Bug & 46 & 70 \\ 
		~ & ~ & ~ & ~ & Bug-regression & 4 & ~ \\ 
		~ & ~ & ~ & ~ & Bug-security & 2 & ~ \\ 
		~ & ~ & ~ & ~ & Feature & 9 & ~ \\ 
		~ & ~ & ~ & ~ & Task & 0 & ~ \\ 
		~ & ~ & ~ & ~ & Unspecified & 9 & ~ \\ \hline
		Chrome & R1 & 1) User participation & - & Bug & 31 & 49 \\ 
		~ & ~ & ~ & ~ & Bug-regression & 6 & ~ \\ 
		~ & ~ & ~ & ~ & Bug-security & 0 & ~ \\ 
		~ & ~ & ~ & ~ & Feature & 11 & ~ \\ 
		~ & ~ & ~ & ~ & Task & 0 & ~ \\ 
		~ & ~ & ~ & ~ & Unspecified & 1 & ~ \\ \hline
		Chrome & R26 & 2) Notice & 2.1) Data subjects & Bug & 20 & 29 \\ 
		~ & ~ & ~ & ~ & Bug-regression & 1 & ~ \\ 
		~ & ~ & ~ & ~ & Bug-security & 0 & ~ \\ 
		~ & ~ & ~ & ~ & Feature & 6 & ~ \\ 
		~ & ~ & ~ & ~ & Task & 1 & ~ \\ 
		~ & ~ & ~ & ~ & Unspecified & 1 & ~ \\ \hline
		Chrome & R41 & 4) Data processing & 4.1) Collection & Bug & 12 & 17 \\ 
		~ & ~ & ~ & ~ & Bug-regression & 1 & ~ \\ 
		~ & ~ & ~ & ~ & Bug-security & 0 & ~ \\ 
		~ & ~ & ~ & ~ & Feature & 3 & ~ \\ 
		~ & ~ & ~ & ~ & Task & 0 & ~ \\ 
		~ & ~ & ~ & ~ & Unspecified & 1 & ~ \\ \hline
	\end{tabular}%
}
\end{table}

\newpage
\begin{table}[ht]
	\centering
	\caption{A table summarising top 10 privacy requirements with their frequency based on issue type in Moodle project.}
	\label{tab:top10-stats-moodle}
	\resizebox{7in}{!}{%
	\begin{tabular}{|l|l|l|l|l|l|l|}
		\hline
		Project & Requirement & Category & Subcategory & Issue type & Frequency & Total frequency \\ \hline
		Moodle & R44 & 1) User participation & - & Bug & 48 & 194 \\ 
		~ & ~ & ~ & ~ & Epic & 0 & ~ \\ 
		~ & ~ & ~ & ~ & Improvement & 44 & ~ \\ 
		~ & ~ & ~ & ~ & New Feature & 57 & ~ \\ 
		~ & ~ & ~ & ~ & Task & 7 & ~ \\ 
		~ & ~ & ~ & ~ & Sub-task & 34 & ~ \\ 
		~ & ~ & ~ & ~ & Functional Test & 4 & ~ \\ \hline
		Moodle & R1 & 1) User participation & - & Bug & 71 & 186 \\ 
		~ & ~ & ~ & ~ & Epic & 1 & ~ \\ 
		~ & ~ & ~ & ~ & Improvement & 44 & ~ \\ 
		~ & ~ & ~ & ~ & New Feature & 59 & ~ \\ 
		~ & ~ & ~ & ~ & Task & 6 & ~ \\ 
		~ & ~ & ~ & ~ & Sub-task & 1 & ~ \\ 
		~ & ~ & ~ & ~ & Functional Test & 4 & ~ \\ \hline
		Moodle & R35 & 3) User desirability & 3.1) Consent & Bug & 29 & 161 \\ 
		~ & ~ & ~ & ~ & Epic & 0 & ~ \\ 
		~ & ~ & ~ & ~ & Improvement & 35 & ~ \\ 
		~ & ~ & ~ & ~ & New Feature & 57 & ~ \\ 
		~ & ~ & ~ & ~ & Task & 4 & ~ \\ 
		~ & ~ & ~ & ~ & Sub-task & 33 & ~ \\ 
		~ & ~ & ~ & ~ & Functional Test & 3 & ~ \\ \hline
		Moodle & R56 & 7) Security & - & Bug & 23 & 150 \\ 
		~ & ~ & ~ & ~ & Epic & 0 & ~ \\ 
		~ & ~ & ~ & ~ & Improvement & 33 & ~ \\ 
		~ & ~ & ~ & ~ & New Feature & 57 & ~ \\ 
		~ & ~ & ~ & ~ & Task & 3 & ~ \\ 
		~ & ~ & ~ & ~ & Sub-task & 34 & ~ \\ 
		~ & ~ & ~ & ~ & Functional Test & 0 & ~ \\ \hline
		Moodle & R38 & 2) Notice & 2.1) Data subjects & Bug & 21 & 112 \\ 
		~ & ~ & ~ & ~ & Epic & 0 & ~ \\ 
		~ & ~ & ~ & ~ & Improvement & 32 & ~ \\ 
		~ & ~ & ~ & ~ & New Feature & 56 & ~ \\ 
		~ & ~ & ~ & ~ & Task & 3 & ~ \\ 
		~ & ~ & ~ & ~ & Sub-task & 0 & ~ \\ 
		~ & ~ & ~ & ~ & Functional Test & 0 & ~ \\ \hline
		Moodle & R42 & 2) Notice & 2.1) Data subjects & Bug & 19 & 109 \\ 
		~ & ~ & ~ & ~ & Epic & 0 & ~ \\ 
		~ & ~ & ~ & ~ & Improvement & 32 & ~ \\ 
		~ & ~ & ~ & ~ & New Feature & 56 & ~ \\ 
		~ & ~ & ~ & ~ & Task & 2 & ~ \\ 
		~ & ~ & ~ & ~ & Sub-task & 0 & ~ \\ 
		~ & ~ & ~ & ~ & Functional Test & 0 & ~ \\ \hline
		Moodle & R34 & 1) User participation & - & Bug & 42 & 57 \\ 
		~ & ~ & ~ & ~ & Epic & 0 & ~ \\ 
		~ & ~ & ~ & ~ & Improvement & 7 & ~ \\ 
		~ & ~ & ~ & ~ & New Feature & 4 & ~ \\ 
		~ & ~ & ~ & ~ & Task & 2 & ~ \\ 
		~ & ~ & ~ & ~ & Sub-task & 0 & ~ \\ 
		~ & ~ & ~ & ~ & Functional Test & 2 & ~ \\ \hline
		Moodle & R26 & 2) Notice & 2.1) Data subjects & Bug & 17 & 43 \\ 
		~ & ~ & ~ & ~ & Epic & 0 & ~ \\ 
		~ & ~ & ~ & ~ & Improvement & 14 & ~ \\ 
		~ & ~ & ~ & ~ & New Feature & 8 & ~ \\ 
		~ & ~ & ~ & ~ & Task & 4 & ~ \\ 
		~ & ~ & ~ & ~ & Sub-task & 0 & ~ \\ 
		~ & ~ & ~ & ~ & Functional Test & 0 & ~ \\ \hline
	\end{tabular}%
}
\end{table}\begin{table}[ht]
\centering
\resizebox{7in}{!}{%
	\begin{tabular}{|l|l|l|l|l|l|l|}
		\hline
		\textbf{Project} & \textbf{Requirement} & \textbf{Category} & \textbf{Subcategory} & \textbf{Issue type} & \textbf{Frequency} & \textbf{Total frequency} \\ \hline
		Moodle & R30 & 2) Notice & 2.1) Data subjects & Bug & 26 & 42 \\ 
		~ & ~ & ~ & ~ & Epic & 0 & ~ \\ 
		~ & ~ & ~ & ~ & Improvement & 11 & ~ \\ 
		~ & ~ & ~ & ~ & New Feature & 1 & ~ \\ 
		~ & ~ & ~ & ~ & Task & 3 & ~ \\ 
		~ & ~ & ~ & ~ & Sub-task & 1 & ~ \\ 
		~ & ~ & ~ & ~ & Functional Test & 0 & ~ \\ \hline
		Moodle & R60 & 7) Security & - & Bug & 27 & 40 \\ 
		~ & ~ & ~ & ~ & Epic & 0 & ~ \\ 
		~ & ~ & ~ & ~ & Improvement & 7 & ~ \\ 
		~ & ~ & ~ & ~ & New Feature & 2 & ~ \\ 
		~ & ~ & ~ & ~ & Task & 4 & ~ \\ 
		~ & ~ & ~ & ~ & Sub-task & 0 & ~ \\ 
		~ & ~ & ~ & ~ & Functional Test & 0 & ~ \\ \hline
	\end{tabular}}
\end{table}\begin{table}[ht]
\centering
\resizebox{7in}{!}{%
	\begin{tabular}{c c c c c c c}
		& & & & & & \\
		& & & & & & \\
		& & & & & & \\
		& & & & & & \\
		& & & & & & \\
		& & & & & & \\
\end{tabular}}
\end{table}

\clearpage
\bibliographystyle{elsarticle-num-names}
\bibliography{privacy-requirements,ICSE2020_kookai_ref}

\end{document}
