\section{Conclusion and future work} \label{sec:conclusion}

%Privacy compliance in software systems has become tremendously significant for organisations that handle personal data of their users. 
In this paper, we have developed a comprehensive taxonomy of privacy requirements based on GDPR, ISO/IEC 29100, Thailand PDPA and APEC privacy framework. Our approach is built upon a content analysis process, adapted from the GBRAM which are generic and applicable to different regulations and privacy standards. We performed reliability assessments and disagreement resolution in the process to ensure that our taxonomy is reliably constructed. Our taxonomy consists of 71 privacy requirements grouped in 7 privacy goal categories. Since the studied regulations and frameworks are not specific to any software types, our taxonomy is generally applicable to a wide range of software applications.

We have also performed a study on how two large projects (Chrome and Moodle) address those privacy requirements in our taxonomy. To do so, we mined the issue reports recorded in those projects and collected 1,374 privacy-related issues. We then classified these issues into our taxonomy through a process which involved multiple coders and the use of IRR assessments and disagreement resolution. We found that the privacy requirements in the user participation category were covered in a majority of the issues. We also found that the time taken to resolve privacy-related issues and the degree of developers' engagement on them were also statistically significantly different from those of non-privacy issues.

Our work lays several important foundations for future research in this area. The systematic method performed in the work enables future research to be conducted on other data protection and privacy regulations and frameworks. The taxonomy can act as a reference for the research community to discuss and expand. We plan to investigate other privacy regulations and policies and extend our taxonomy, if necessary. The requirements in our taxonomy are written in natural language and structured into templates. Although we believe that this is the most intuitive form to developers, future work could explore other alternative forms such as semantic frame-based representation \cite{Bhatia2019}. We have manually derived requirements in this study as it is essential to examine structure of statements and how privacy requirements are expressed in different regulations and frameworks. However, the requirements extraction process can be automated using NLP techniques (e.g. \cite{Zeni2015} and \cite{Sleimi2018}).

Future work also involves exploring how issue reports in other projects (e.g. health and mobile applications) attend to the requirements in our taxonomy. This also includes other issues that are not tagged as privacy-related. In addition, software developers would be able to participate in validating the use of taxonomy and their understanding towards the derived privacy requirements. Legal experts could also be involved to help interpret legal perspective of the taxonomy. Furthermore, we plan to develop tool support to automate the privacy requirements identification and classification tasks when users report issues in ITS. Other potential future work includes the investigation of the use of the taxonomy with semantic web technologies to facilitate computational and regulatory risk analysis purposes. The investigation of other forms of traceability relationships such as the traceability between code and issue reports or code and privacy requirements can also be further studied. 

%Privacy compliance in software systems has become tremendously significant for organisations that handle personal data of their users. In this paper, we have developed a comprehensive taxonomy of privacy requirements based on four well-known standardised data protection regulations and privacy frameworks (GDPR, ISO/IEC 29100, Thailand PDPA and APEC privacy framework). Our approach is built upon a content analysis process which is adapted from the Goal-Based Requirements Analysis Method and based on Grounded Theory \cite{Antn2004}. Thus, the steps in this approach (privacy requirements identification, refinement and classification) are generic and applicable to different regulations and privacy standards. %In our approach, privacy requirements are formed by three components: actions, affected parties/objects and target results. These are commonly found in the narrative statements across different regulations. 

%We performed reliability assessments and disagreement resolution in the process to ensure that our taxonomy is reliably constructed. Our taxonomy consists of 71 privacy requirements grouped in 7 privacy goal categories. Since the studied regulations and frameworks are not specific to any software types, our taxonomy is generally applicable to a wide range of software applications. In this study, we retain the privacy requirements at this level since the implementation can vary depending on organisational structures, software architectures, existing implementation and knowledge, and the aptitude of software development teams. Thus, our taxonomy focuses on defining privacy requirements a software needs to satisfy, leaving how the software is designed and implemented to meet those requirements to project stakeholders.

%We have also performed a study on how two large projects (Google Chrome and Moodle) address those privacy requirements in our taxonomy. To do so, we mined the issue reports recorded in those projects and collected 1,374 privacy-related issues. We then classified these issues into our taxonomy through a process which involved multiple coders and the use of inter-rater reliability assessments and disagreement resolution. We found that the privacy requirements in the user participation category were covered in a majority of the issues, while none of the issues were found to address the breach category. The taxonomy also captures all the privacy requirements in the issue reports. We also found that the time taken to resolve privacy-related issues and the degree of developers' engagement on them were also statistically significantly different from those of non-privacy issues. %We note that the taxonomy can be applied to software requirements and user stories. In fact, there are issue reports in the dataset which are user stories or requirements for new features.

%We believe that the taxonomy and the dataset are important contributions to the community (given that none exists). This would initiate discussions and further work (e.g. refining and/or extending the taxonomy, performing user studies, etc.) in the community. 

%to identify actors, actions, objects and other relevant constraints from text documents (i.e. regulations and frameworks). These methods have been used in several studies (e.g. \cite{Zeni2015} and \cite{Sleimi2018}). Legal experts could also be involved to help interpret legal perspective of the taxonomy in the future work.