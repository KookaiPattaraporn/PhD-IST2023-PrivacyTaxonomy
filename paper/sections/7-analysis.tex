\section{Analysis and discussions} \label{sec:results}

The study presented in the previous section generates not only a valuable dataset but also important insights into how privacy requirements have been addressed in Chrome and Moodle issue reports. In this section, we discuss and analyse some of the key findings and implications.

%This study not only provides us with a labelled dataset (which can be used for future research in this area) but also a number of insights in the two projects. We discuss here some important findings:

\subsection{The top and least concerned privacy requirements}
%We first present how frequent the issues are classified into the privacy requirements in each privacy goal in both projects in the percentage of occurrences by requirements column (see Table \ref{tab:table2}).

Table \ref{tab:req-occurrences} presents the coverage of each category in Chrome and Moodle issue reports. \newtext{We found 1,157 and 1,275 privacy requirements in Chrome and Moodle issue reports respectively (2,432 privacy requirements in total).} In both projects, the majority of the issues address the user participation requirements (Category 1). Most issue reports in Moodle address more than one privacy requirement across different privacy goal categories. This results in Moodle having higher coverage (in terms of the occurrences) than Chrome. \newtext{It is worth noting that requirements R1 ALLOW the data subjects to access and review their personal data, R26 PROVIDE the data subjects the information relating to the policies, procedures, practices and logic of the processing of personal data, R30 PROVIDE the data subjects the information relating to the processing of personal data with standardised icons, R44 ALLOW the data subjects to erase personal data and R60 IMPLEMENT appropriate technical and organisational measures to protect personal data were in the top 10 in both projects.}

\begin{table}[ht]
	\centering
	\caption{\newtext{The number of privacy requirements found in Chrome and Moodle issue reports categorised by category.}}
	\label{tab:req-occurrences}
	\resizebox{3.5in}{!}{%
		\begin{tabular}{@{}l c c@{}}
			\toprule
			\multicolumn{1}{c}{\multirow{2}{*}{\textbf{Category}}} & \multicolumn{2}{c}{\textbf{\begin{tabular}[c]{@{}c@{}}No. of mined privacy requirements \\ in each project by category\end{tabular}}} \\ \cmidrule(l){2-3}
			\multicolumn{1}{c}{}                                   & \multicolumn{1}{c}{\textbf{Google Chrome}}      & \multicolumn{1}{c}{\textbf{Moodle}}      \\ \midrule
			1) User participation							&   321                                              &   328                                       \\
			2) Notice                                              &   272                                              &   229                                       \\
			3) User desirability							&    241                                             &    194                                      \\
			4) Data processing 							&     117                                            &  55                                        \\
			5) Breach 											&    0                                             &     0                                     \\
			6) Complaint/Request 						&   1                                              &   9                                       \\
			7) Security 										&    160                                             &   214                                       \\ \bottomrule
		\end{tabular}%
	}
\end{table}

\begin{comment}
\begin{table}[ht]
	\centering
	\caption{The number of privacy requirements in each category and the percentage of occurrences in Chrome and Moodle issue reports}
	\label{tab:req-occurrences}
	\resizebox{3.25in}{!}{%
		\begin{tabular}{p{3cm} c c c}
			\toprule
			\multicolumn{1}{c}{\multirow{2}{*}{\textbf{Privacy goals}}} & \multicolumn{1}{c}{\multirow{2}{*}{\begin{tabular}[c]{@{}c@{}}\textbf{No. of privacy }\\\textbf{requirements~ }\end{tabular}}} & \multicolumn{2}{c}{\begin{tabular}[c]{@{}c@{}}\textbf{Percentage of}\\\textbf{occurrences (\%) }\end{tabular}} \\
			\cline{3-4}
			\multicolumn{1}{c}{} & \multicolumn{1}{c}{} & \multicolumn{1}{c}{\begin{tabular}[c]{@{}c@{}}\textbf{Google }\\\textbf{Chrome}\end{tabular}} & \multicolumn{1}{c}{\textbf{Moodle}} \\
			\midrule
			User participation & 9 & 35.83 & 68.62\\
			Notice & 32 & 30.36 & 47.91\\
			User desirability & 10 & 28.01 & 40.59\\
			Data processing & 16 & 13.06 & 11.51\\
			Breach & 6 & 0.00 & 0.00\\
			Complaint/Request & 5 & 0.11 & 1.88\\
			Security & 13 & 17.86 & 44.77\\
			\hline
			\multicolumn{4}{p{8.5cm}}{\footnotesize *Since one issue can relate to multiple privacy requirements, the sum of percentage exceeds 100\%} \setlength\lineskip{0pt}
		\end{tabular}%
	}
\end{table}
\end{comment}

The top three most concerned requirements in Chrome are R30, R44 and R60 (see Figure \ref{fig:top10} and \newtext{Table \ref{tab:top10-detail}}). Note that these requirements belong to three different privacy goal categories (refer to Table \ref{tab:sample-requirements} for details of the privacy goals and requirements we discussed here). The top three requirements covered in Moodle issues are R44, R1 and R35 \newtext{OBTAIN the opt-in consent for the processing of personal data for specific purposes}. %It is also worth noting that requirements R1, R26 \newtext{PROVIDE the data subjects the information relating to the policies, procedures, practices and logic of the processing of personal data}, R30, R44 and R60 are in the top 10 in both projects.

Requirement R44 was in the top three most concerned requirements in both projects, suggesting that allowing the data subjects to erase their personal data is a highly important privacy requirement for both Chrome and Moodle. Requirements R30 and R36 were also addressed in many privacy-related issues in Chrome. This suggests that providing information with standardised, visible and meaningful icons which inform the intended processing of personal data for users is an important privacy concern in Chrome (R30). In addition, many issues in Chrome also focus on addressing the privacy requirement that users are presented with all available options related to the processing of personal data (R36).

Apart from requirement R44, the other two requirements most frequently covered in Moodle issue reports are R1 and R35 (note that they are different from those in Chrome). Approximately 39\% issues in Moodle are related to requirement R1. This implies that Moodle has a strong emphasis on allowing users to access their personal data such as grade records, course participation and course enrolment records. This function is not only important for Moodle, but also in Chrome (R1 is also in the top 10 for Chrome). User consent is a major concern in privacy protection. We found that a large number of issue reports in Moodle explicitly requires the system to obtain consent from users for processing personal data based on specific purposes (R35).

%The common requirement R26 emphasises on the importance to provide information related the processing of personal data to users in terms of documentation. Both systems are also required to implement appropriate mechanisms to protect personal data. All the requirements here potentially show a set of key functions that the stakeholders of software systems should consider when handling personal data.

It is interesting to note that none of the requirements in the breach category were found in both Chrome and Moodle as they were not directly observed from issue reports or recorded in the ITS. These goals can be evidenced through high-level organisational activities such as Data Protection Impact Assessment, Legitimate Interest Assessment and breach notifications. Our future work will investigate this further.

\begin{figure}[ht]
	\centering
	\includegraphics[width=1\linewidth]{Figures/"Top10-occ-by-type"}
	\caption{Top 10 privacy requirements occurrences in Google Chrome and Moodle datasets categorised by issue types}
	\label{fig:top10}
\end{figure}

\begin{table}[ht]
	\centering
	\caption{\newtext{A summary of top 10 concerned privacy requirements in Google Chrome and Moodle datasets.}}
	\label{tab:top10-detail}
	\resizebox{3.5in}{!}{%
	\begin{tabular}{@{}ccllc@{}}
		\toprule
		\textbf{Project}                                      & \textbf{\begin{tabular}[c]{@{}c@{}}Req-\\ uirement\end{tabular}} & \multicolumn{1}{c}{\textbf{Category}} & \multicolumn{1}{c}{\textbf{Subcategory}}                               & \textbf{\begin{tabular}[c]{@{}c@{}}Frequency\\ (Occurrences)\end{tabular}} \\ \midrule
		\multirow{10}{*}{\textbf{Chrome}}                     & R30                  & 2) Notice                             & 2.1) Data subjects                                                     & 209                                                                        \\
		& R44                  & 1) User participation                 & -                                                                      & 204                                                                        \\
		& R60                  & 7) Security                           & -                                                                      & 135                                                                        \\
		& R8                   & 3) User desirability                  & \begin{tabular}[c]{@{}l@{}}3.1) Consent\\ 3.3) Preference\end{tabular} & 144                                                                       \\
		& R36                  & 3) User desirability                  & 3.2) Choice                                                            & 119                                                                        \\
		& R45                  & 1) User participation                 & -                                                                      & 73                                                                         \\
		& R53                  & 4) Data processing                    & 4.4) Erasure                                                           & 70                                                                         \\
		& R1                   & 1) User participation                 & -                                                                      & 49                                                                         \\
		& R26                  & 2) Notice                             & 2.1) Data subjects                                                     & 29                                                                         \\
		& R41                  & 4) Data processing                    & 4.1) Collection                                                        & 17                                                                         \\ \midrule
		\multicolumn{1}{l}{\multirow{10}{*}{\textbf{Moodle}}} & R44                  & 1) User participation                 & -                                                                      & 194                                                                        \\
		\multicolumn{1}{l}{}                                  & R1                   & 1) User participation                 & -                                                                      & 186                                                                        \\
		\multicolumn{1}{l}{}                                  & R35                  & 3) User desirability                  & 3.1) Consent                                                           & 161                                                                        \\
		\multicolumn{1}{l}{}                                  & R56                  & 7) Security                           &  -                                                                      & 150                                                                        \\
		\multicolumn{1}{l}{}                                  & R38                  & 2) Notice                             & 2.1) Data subjects                                                     & 112                                                                        \\
		\multicolumn{1}{l}{}                                  & R42                  & 2) Notice                             & 2.1) Data subjects                                                     & 109                                                                        \\
		\multicolumn{1}{l}{}                                  & R34                  & 1) User participation                 & -                                                                      & 57                                                                         \\
		\multicolumn{1}{l}{}                                  & R26                  & 2) Notice                             & 2.1) Data subjects                                                     & 43                                                                         \\
		\multicolumn{1}{l}{}                                  & R30                  & 2) Notice                             & 2.1) Data subjects                                                     & 42                                                                         \\
		\multicolumn{1}{l}{}                                  & R60                  & 7) Security                           & -                                                                      & 40                                                                         \\ \bottomrule
	\end{tabular}%
}
\end{table}

\begin{comment}
	\begin{figure}
		\centering
		\includegraphics[width=1\linewidth]{Figures/"Top10-occ"}
		\caption{Top 10 privacy requirements occurrences in Google Chrome and Moodle datasets}
		\label{fig:top10}
	\end{figure}
\end{comment}

%It is interesting to note that all the seven requirements in the accountability goal category were not covered in both projects. These requirements are relevant to privacy breach notification and compliance management. They mainly focus on the tasks that are under management responsibility of the controllers (e.g. R66 and R67). This suggests three possible scenarios: (i) the existing systems did not provide any functions regarding these requirements; (ii) these circumstances might not have been recorded in the issue trackers; or (iii) the privacy requirements have already been met at the implementation level. Our future work will investigate this further.

We have performed further analysis on the issue reports that were classified into the top concerned requirements in Chrome and Moodle datasets. We analyse four factors focusing on how the contributors treat those issue reports: issue types, the time took to resolve an issue, the number of contributors involved and the number of comments associated with the issue.

In Chrome dataset, issue reports have five different types: bug, bug-regression, bug-security, feature and task. Bug type reports malfunctioning functionalities in current version of the system. Bug-regression focuses on the functions that used to work correctly in the previous versions, but are broken in the current version. Bug-security reports malfunctions that are risky to user security. Feature type requests for an implementation of a new function/feature. Task type, which is not a bug or feature, defines a piece of work that needs to be completed for an issue. A small group of issues did not have their issue types specified. From our study, we found that the issue reports whose issue type is a bug are the largest group in every top concerned privacy requirement \newtext{(see Table \ref{tab:issue-type})}. These bug issue reports as well as bug-regression and bug-security took less time to resolve comparing to feature request issue reports on average for all the top concerned privacy requirements. In addition, the bug issue reports usually involved with a smaller number of contributors and had less discussions than the feature request issue reports. We have also investigated the discussions of bug issue reports classified into the top concerned requirements. We found that the bug issues were fixed after fifteen comments. However, the discussions of feature issues contain more details (e.g. use case scenarios, discussion points, screenshots and code snippets) than the bug issues.

There are seven different issue types in Moodle dataset: bug, epic, improvement, new feature, task, sub-task and functional test. The definitions of bug, new feature, task and sub-task issue types are similar to those mentioned in Chrome dataset. Epic issue type collects a group of issues that needs to be completed over a period of time. An improvement issue type is an enhancement to an existing feature. Functional test type contains the information and steps used for testing a particular function. From the analysis in Moodle dataset, the bug issue type contains the largest number of issue reports, followed by the feature issue type. We found that the bug issues did not only report the malfunctions, but they also reported the missing functionalities in the system (e.g. implement core\_privacy for block rss client). It is interesting to note that the new feature issues took less time to resolve comparing to the bug issues on average for all the requirements except R42. However, both issue types have similar number of comments (i.e. 11 to 12 comments) and number of contributors (i.e. 5).

\begin{table}[h]
	\centering
	\caption{\newtext{Number of issue reports in Chrome and Moodle projects counted by issue type.}}
	\label{tab:issue-type}
	\resizebox{3.5in}{!}{
	\begin{tabular}{p{2.5cm} p{1cm} p{2.5cm} p{1cm}}
		\toprule
		\textbf{Chrome} & \textbf{\#issues} & \textbf{Moodle} & \textbf{\#issues} \\
		\midrule
		Bug & 620 & Bug & 223 \\
		Bug-regression & 59 & Epic & 3 \\
		Bug-security & 36 & Improvement & 101 \\
		Feature & 132 & New Feature & 75 \\
		Task & 5 & Task & 26 \\
		Unspecified & 44 & Sub-task & 37 \\
		~ & ~ & Functional test & 13 \\
		\bottomrule
		\textbf{Total} & \textbf{896} & \textbf{Total} & \textbf{478} \\
		\bottomrule
	\end{tabular}%
	}
\end{table}

\newtext{We have summarised and presented the number of mined privacy requirements by category and by top requirements in both Chrome and Moodle projects\footnote{A full analysis can be found in the replication package \cite{reppkg-pridp}.}. These categories represent a group of activities/functionalities related to privacy in the projects while the privacy requirements identify specific needs expected to be fulfilled in a system. From our analysis, it presented how frequent those activities were concerned in those projects. This result allows the contributors to carefully consider the privacy-related functionalities in their projects. Software engineers can easily use the privacy requirements in the taxonomy to identify specific functionalities that were malfunctioned (e.g. bugs), needed to be implemented (e.g. feature requests) or needed to be changed (e.g. change requests) in issue reports. They can be confident that those functionalities are required by the data protection and privacy regulations, standards and frameworks. The privacy requirements mapped to each issue report can be also used as privacy measures. The contributors can assess whether the privacy measures are passed or failed. If the privacy requirements are not implemented or not properly functioned, then this issue report is failed for this project. On the other hand, if the privacy requirements are implemented or properly worked, then this issue report is passed. However, we did the analysis with the fixed issue reports, thus we could not assess privacy measures in those projects.}

\newtext{The top 10 privacy requirements in Chrome project covered 5 categories which are user participation, user desirability, notice, security and data processing (sorted by frequency). This set of privacy requirements covered only the data subjects subcategory in the notice category and collection and erasure subcategories in the data processing category. However, all the subcategories in user desirability category were all concerned. Most of the issue reports associated with the top 10 privacy requirements were bug, followed by feature issue type. This implies that Chrome had not properly implemented the functionalities related to providing users the way to execute their individual rights, obtaining and managing user consent and preferences and providing notice to users.}

\newtext{Four categories including user participation, notice, security and user desirability were covered by the top 10 privacy requirements in Moodle. The data subjects and consent subcategories in the notice and user desirability categories were covered by this set of privacy requirements respectively. Unlike Chrome, the majority of the issue reports associated with the top 10 privacy requirements were feature request, followed by bug type. Moodle focused on implementing new features that allow users to execute their individual rights and inform users of relevant privacy-related information. In addition, Moodle also emphasised on implementing security measures to protect personal data.}

%\textbf{RQ2: Were privacy and non-privacy issues treated differently?} \\

\subsection{The treatment of privacy and non-privacy issues}

We investigate if privacy issues were treated differently from non-privacy issues in Chrome and Moodle. We focus on observing two kinds of treatments: the time it took to resolve an issue and the number of comments associated with the issue. The former reflects how fast an issue was resolved while the latter indicates the attention and engagement of the project team to the issue. We randomly sampled the dataset we built earlier using a 95\% confidence level with a confidence interval of 5\footnote{https://www.surveysystem.com/sscalc.htm} to obtain 269 privacy issues from Chrome and 213 from Moodle. Applying the same sampling scheme, we randomly selected 382 non-privacy issues from Chrome and 380 from Moodle - these issues were not tagged as privacy in the ``component'' field. Note that the resolution time is calculated from the number of days between reported date and the date when the issue was flagged as being resolved.

\begin{table}
	\centering
	\caption{Results of the Wilcoxon rank-sum test: non-privacy vs. privacy issues}
	\label{tab:ranksum}
	\resizebox{8.5cm}{!}{
		\begin{tabular}{l l l l l}
			\toprule
			\textbf{Project} & \textbf{Attribute} & \textbf{One-sided tail} & \textbf{p-value} & \textbf{Effect size}\\
			\midrule
			Google Chrome & Resolution time & Less & $<$0.001 & 0.578 \\
			Google Chrome & \#Comments & Less & $<$0.001 & 0.691 \\
			Moodle & Resolution time & Greater & $<$0.001 & 0.609 \\
			Moodle & \#Comments & Greater & $<$0.001 & 0.604\\
			\bottomrule
		\end{tabular}%
	}	
\end{table}

We employ the Wilcoxon rank-sum test (also known as Mann-Whitney U test), a non-parametric hypothesis test which compares the difference between two independent observations \cite{Wild1997}. We performed two tests between privacy and non-privacy samples, one for the resolution time and the other for the number of comments. The results (see Table \ref{tab:ranksum}) show that the resolution time and the number of comments are statistically significantly (\textit{p-value $\leq$ 0.001}) different between privacy and non-privacy issues in both Chrome and Moodle with effect size greater than 0.5 in all cases.

We also compare the median rank of the two samples using one-tailed test. Our results show that privacy issues were resolved more quickly and attracted less comments than non-privacy issues in Moodle (see Table \ref{tab:ranksum}). On the other hand, it took longer to resolve privacy issues than non-privacy issues in Chrome. Also, privacy issues in Chrome tend to attract more discussion than non-privacy issues. 

%We observed that the contributors in Chrome were confident and had more experience in resolving non-privacy issues. Hence, these issues attracted less discussion and was resolved more quickly. By contrast, privacy issues in Chrome attracted more discussions since the contributors were uncertain about the issues and their affected components in the system. We observed five examples that the contributors commented in those privacy issues as follows: (i) the contributors did not know what the affected components reported in the issues do (e.g. issue 345741); (ii) the contributors could not identify the causes of issues; (iii) the contributors required time and effort to come up with potential solutions; (iv) the contributors needed to assess the difficulties of the issues and their resolutions; and (v) the contributors did not know whom to assign the work. These reasons also led to longer time to resolve the privacy issues in Chrome. In addition, the Chrome project does not have a well-defined process that specifically handles privacy. Hence, the contributors need to ensure that fixing privacy issues will not create another problem in different components. Thus, resolving privacy issues attracted a lot of discussions, leading to longer resolution time.

\newtext{We observed the following patterns when we went through the comments in non-privacy issue reports in Chrome: (i) the issue reports were assigned to relevant contributors and got resolved without any discussion (e.g. Issue 142322\footnote{https://bugs.chromium.org/p/chromium/issues/detail?id=142322)}, (ii) the contributors asked if fixing those issue reports did not affect other parts without discussing on how to fix the issues (e.g. Issue 914196\footnote{https://bugs.chromium.org/p/chromium/issues/detail?id=914196}) and (iii) the contributors discussed on workarounds meaning that they knew how to fix the issues but direct method could not be used (e.g. Issue 1082077\footnote{https://bugs.chromium.org/p/chromium/issues/detail?id=1082077}). These patterns show that the contributors had less discussion with regard to identifying root causes or solutions of the issues. Based on these findings, the contributors in Chrome tended to be confident and had more experience in resolving non-privacy issues.} 

By contrast, privacy issues in Chrome attracted more discussions since the contributors were uncertain about the issues and their affected components in the system. We observed five examples that the contributors commented in those privacy issues as follows: (i) the contributors did not know what the affected components reported in the issues do (e.g. issue 345741); (ii) the contributors could not identify the causes of issues; (iii) the contributors required time and effort to come up with potential solutions; (iv) the contributors needed to assess the difficulties of the issues and their resolutions; and (v) the contributors did not know whom to assign the work. These reasons also led to longer time to resolve the privacy issues in Chrome. In addition, the Chrome project does not have a well-defined process that specifically handles privacy. Hence, the contributors need to ensure that fixing privacy issues will not create another problem in different components. Thus, resolving privacy issues attracted a lot of discussions, leading to longer resolution time.

On the other hand, privacy issues in Moodle were resolved more quickly and attracted less comments than non-privacy issues. We observe that the privacy issues were well reported and clearly explained in Moodle. Moodle contributors were familiar with privacy-related functionalities and relevant system components. In addition, Moodle has a clearly defined infrastructure to handle privacy and privacy compliance in the system (e.g. privacy API \cite{Moodle2019} and GDPR for plugin developers \cite{Nicols2018}). This infrastructure includes a number of components that support privacy-related functionalities and several key individual rights in GDPR (e.g. accessing to personal data and requesting for deletion). When there is a privacy-related bug or new feature request, the contributors can consult the privacy API documentation and identify the components that they must fix or implement. Hence, the privacy issues took less time and attracted less comments in Moodle.

%Moodle has a clear infrastructure of privacy compliance (e.g. privacy API \cite{Moodle2019}). Hence, we observe that Moodle developers did not need much discussion before resolving privacy issues, resulting in shorter resolution time. We observe that privacy issues in Chrome tend to affect many different components in the system. Thus, resolving those issues attracted a lot of discussions, leading to longer resolution time.

%In contrast, privacy issue reports use less time to fix in Moodle. Moodle's non-privacy issue reports have more interactions than privacy-related issues. Since Moodle has clearly defined its infrastructure of privacy compliance (e.g. privacy API \cite{Moodle2019}), the developers do not need to discuss a lot before resolving issues. This also reflects the duration they use to resolve issue reports.

%Based on the results, they show some different aspects of how privacy and non-privacy are treated in different open-source software projects by the fix time and number of comments.

%To answer this question, we initially investigated two issue report attributes (fix time and number of comments) between privacy and non-privacy issues of Google Chrome and Moodle. We set our scope to not observe correlations between other attributes as most bug report attributes are not correlated with bug-fix time \cite{Bhattacharya2011}. Fix time (also represents development time for feature requests issue type) reflects how quickly the developers intend to fix privacy and non-privacy issue reports. The number of comments possibly indicates the attention and engagement of relevant stakeholders to discuss on issue reports.

%We randomly sampled issue reports (with their status as fixed) using a 95\% confidence level with a confidence interval of 5\footnote{https://www.surveysystem.com/sscalc.htm}. There were 269 and 213 issue reports sampled from Google Chrome and Moodle datasets used in the issue reports classification step. Applying the same scheme, we randomly selected 382 and 213 issue reports from Google Chrome and Moodle non-privacy issue populations (i.e. they are not tagged as privacy in the ``component'' field). The fix time is calculated from the difference between reported date and fixed date. In this study we use \textit{the number of days} as a unit. The number of comments is straightforward.

%Firstly, we employ the Wilcoxon rank-sum test (also known as Mann-Whitney U test), a non-parametric hypothesis test which compare the difference between two independent observations \cite{Wild1997}. We performed two tests between privacy and non-privacy samples of both datasets based on each attribute. This step tests whether there is a statistically significant difference between two groups of issue reports for each attribute. The results show that fix time and number of comments are statistically significant (\textit{p-value $\leq$ 0.005}) between privacy and non-privacy issue reports for both projects.

%We then performed further analysis on each attribute for each project to compare the median rank of these two samples using one-tailed test. Our results show that all the attributes provide statistical significance between privacy and non-privacy issue reports for both datasets (see Figure \ref{tab:ranksum}). We also report the effect size of the results \cite{McGraw1992}. In Google Chrome, non-privacy issue reports are likely to use less fix time comparing to privacy-related issue reports. The number of comments in non-privacy issue reports are smaller than privacy-related issues. From the observation, privacy issues in Google Chrome tend to affect different components in its system. However, with a number of developers are contributed to the Google Chrome project, it possibly affects the fix time.

%In contrast, privacy issue reports use less time to fix in Moodle. Moodle's non-privacy issue reports have more interactions than privacy-related issues. Since Moodle has clearly defined its infrastructure of privacy compliance (e.g. privacy API \cite{Moodle2019}), the developers do not need to discuss a lot before resolving issues. This also reflects the duration they use to resolve issue reports.

%Based on the results, they show some different aspects of how privacy and non-privacy are treated in different open-source software projects by the fix time and number of comments.

%\todo{check what should be the appropriate digits used to report results}


\begin{comment}

Note:
p-value for fix time = 0.000058, comment = 0.0001
I think the interpretation from here is that if we set the null hypothesis (H0) as 'the fixing time of privacy and non-privacy issues are similar'. And H1 as 'the fixing time of privacy and non-privacy issues are different'. Then, the result from p-value (which is <0.05) rejects H0. Hence, the fixing time between these two groups are statistically significant difference, agree?

There are two values of tail: greater and less.
Now, my parameters fed into the function is non-privacy (as x) and privacy (as y) values. Greater means the median of x is higher than y. On the other hand, less means the median of x is less than y. The p-value of this test is also reported. All of our tests are statistically significant based on the p-value reported in each test.

The result of pq ranksums test - moodle - fixing time - one-sided:
U-val     tail     p-val      RBC      CLES
MWU  27793.5  greater  0.000029 -0.22522  0.604157

The result of pq ranksums test - chrome - fixing time - one-sided:
U-val  tail     p-val       RBC      CLES
MWU  42663.0  less  0.000112  0.169641  0.578077

The result of pq ranksums test - moodle - comments - one-sided:
U-val     tail    p-val      RBC      CLES
MWU  27624.5  greater  0.00005 -0.21777  0.591086

The result of pq ranksums test - chrome - comments - one-sided:
U-val  tail         p-val       RBC      CLES
MWU  29894.0  less  4.294377e-20  0.418167  0.690915
\end{comment}


