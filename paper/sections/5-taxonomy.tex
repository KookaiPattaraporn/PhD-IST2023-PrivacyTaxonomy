\section{\newtext{Privacy Requirements Taxonomy}} \label{sec:taxonomy}

This section presents a taxonomy of privacy requirements that we have developed based on the GDPR, ISO/IEC 29100, Thailand PDPA and APEC privacy framework. Our taxonomy consists of a comprehensive set of 71 privacy requirements classified into 7 categories. The full version of the taxonomy can be found in online Annex \cite{reppkg-pridp}. We now highlight some of the important requirements in each category (see Table \ref{tab:sample-requirements}). We note that there are typically four types of roles involved in a privacy requirement: (i) data subjects who provide their personal data for processing, give consent and determine their privacy preferences; (ii) data controllers who determine what data to be collected and the purpose of personal data collection and processing; (iii) data processors who process the personal data corresponding to the specified purpose and (iv) third parties who in case receive personal data from the controllers or processors.

\vspace{-4mm}

\begin{table}[htbp]
	\caption{Selected privacy requirements that are referred in the paper. The full taxonomy is available in the replication package \cite{reppkg-pridp}.}
	\label{tab:sample-requirements}
	\small
	\begin{tabular}{p{8.5cm}}
		\toprule % <-- Toprule here
		\textbf{Privacy requirements}\\
		\midrule % <-- Midrule here
		
		\textbf{Category 1: User participation} \\
		R1 ALLOW the data subjects to access and review their personal data \\
		R6 ALLOW the data subjects to withdraw consent \\
		R34 ALLOW the data subjects to obtain and reuse their personal data for their own purposes across different services \\
		R44 ALLOW the data subjects to erase their personal data     \\
		R45 ALLOW the data subjects to rectify their personal data \\
		
		\vspace{1mm}
		
		\textbf{Category 2: Notice} \\
		\newtext{\textbf{Subcategory 2.1: Data subjects}} \\
		R12 INFORM the data subjects the reason(s) for not taking action on their request and the possibility of lodging a complaint \\
		R15 NOTIFY the data subjects the data breach which is likely to result in high risk \\
		R19 PROVIDE the data subjects an option to choose whether or not to provide their personal data \\
		R22 PROVIDE the data subjects with the identity and contact details of a controller/controller's representative \\
		R26 PROVIDE the data subjects the information relating to the policies, procedures, practices and logic of the processing of personal data   \\
		R27 PROVIDE the data subjects the recipients/categories of recipients of their personal data   \\
		R30 PROVIDE the data subjects the information relating to the processing of personal data with standardised icons \\
		R38 PROVIDE the data subjects the purpose(s) of the collection of personal data      \\
		R39 PROVIDE the data subjects the purpose(s) of the processing of personal data \\
		R42 PROVIDE the data subjects the categories of personal data concerned  \\			
		R55 PROVIDE the data subjects the period/criteria used to store their data \\
		
		\newtext{\textbf{Subcategory 2.2: Relevant parties}} \\
		R17 SHOW the relevant stakeholders the consent given by the data subjects to process their personal data  \\
		R66 NOTIFY a supervisory authority the data breach   \\
		%R67 NOTIFY relevant privacy stakeholders about a data breach  \\
		
		\vspace{1mm}
		
		\textbf{Category 3: User desirability} \\
		\newtext{\textbf{Subcategory 3.1: Consent}} \\
		R6 ALLOW the data subjects to withdraw consent \\
		R8 IMPLEMENT the data subject's preferences as expressed in his/her consent \\
		R35 OBTAIN the opt-in consent for the processing of personal data for specific purposes \\
		R47 ERASE the personal data when a consent is withdrawn \\
		
		\newtext{\textbf{Subcategory 3.2: Choice}} \\
		R19 PROVIDE the data subjects an option to choose whether or not to provide their personal data  \\
		R36 PRESENT the data subjects an option whether or not to allow the processing of personal data \\
		
		\newtext{\textbf{Subcategory 3.3: Preference}} \\
		R8 IMPLEMENT the data subject's preferences as expressed in his/her consent \\
		
		\vspace{1mm}
		
		\textbf{Category 4: Data processing} \\	
		\newtext{\textbf{Subcategory 4.1: Collection}} \\		
		R41 COLLECT the personal data as necessary for specific purposes   \\

		\bottomrule
		
	\end{tabular}
\end{table}

\begin{table}[h]
	%\caption{Selected privacy requirements that are referred in the paper. The full taxonomy is available in the replication package \cite{replicationpkg}.}
	\label{tab:sample-requirements-2}
	\small
	\begin{tabular}{p{8.5cm}}
		\toprule % <-- Toprule here
		\textbf{Privacy requirements (Continued)}\\
		\midrule % <-- Midrule here
		
		%\textbf{Category 4: Data processing (Continued)} \\	
		
		\newtext{\textbf{Subcategory 4.2: Use}} \\		
		R40 USE the personal data as necessary for specific purposes specified by the controller \\
		
		\newtext{\textbf{Subcategory 4.3: Storage}} \\		
		R43 STORE the personal data as necessary for specific purposes \\
		
		\newtext{\textbf{Subcategory 4.4: Erasure}} \\		
		R7 ERASE the personal data when it has been unlawfully processed \\
		R46 ERASE the personal data when the data subjects object to the processing \\
		R47 ERASE the personal data when a consent is withdrawn \\
		R52 ERASE the personal data when it is no longer necessary for the specified purpose(s)   \\
		R53 ERASE the personal data when the purpose for the processing has expired  \\
		
		\newtext{\textbf{Subcategory 4.6: Record}} \\		
		R13 MAINTAIN a record of personal data processing activities \\
		
		\vspace{1mm}
		
		\textbf{Category 5: Breach} \\
		R15 NOTIFY the data subjects the data breach which is likely to result in high risk \\
		R66 NOTIFY a supervisory authority the data breach   \\
		R67 NOTIFY relevant privacy stakeholders about a data breach \\
		
		\vspace{1mm}
		
		\textbf{Category 6: Complaint/Request} \\
		R12 INFORM the data subjects the reason(s) for not taking action on their request and the possibility of lodging a complaint \\
		R31 REQUEST the data subjects the additional information necessary to confirm their identity when making a request relating to the processing of personal data \\
		
		\vspace{1mm}
		
		\textbf{Category 7: Security} \\
		R56 ALLOW the authorised stakeholders to access personal data as instructed by a controller \\
		R60 IMPLEMENT appropriate technical and organisational measures to protect personal data \\
		R63 PROTECT the personal data from unauthorised access and processing \\
		R65 IMPLEMENT a function to comply with local requirements and cross-border transfers \\
		\bottomrule
	\end{tabular}
\end{table}

%\vspace{-20mm}

\newtext{\subsection{User participation}}
\newtext{All the requirements in this category specify the functionalities provided for data subjects to execute their individual rights in managing their personal data. The data subjects must be able to access and review, erase and rectify their personal data (e.g. R1, R44 and R45). The systems must allow the data subjects to withdraw consent (R6). The systems must also provide the data subjects their personal data when they would like to obtain and reuse their personal data for their own purposes across different services (R34). The controllers shall allow the data subjects to object to and restrict the processing of their personal data (R3 and R4). The data subjects must be able to withdraw consent (R6) or lodge a complaint to a supervisory authority (R2).}

\newtext{\subsection{Notice}}
\newtext{This category is the largest group consisting of 32 privacy requirements in the taxonomy. It consists of two sub-categories: data subjects and relevant parties. Most of the requirements in this category are concerned with the transparency of personal data processing (e.g. R17, R22 and R42). It has a set of requirements for the data subjects to be informed and/or notified of relevant information and individual rights related to the processing of personal data. Those requirements aim to ensure that a system shall provide information related to the processing of personal data (e.g. what personal data is required, who is responsible for their data and results from requests) to data subjects. The information also includes privacy policies, procedures, practices and logic of the processing of personal data. Personal data shall not be misused, and the data subjects have the right to know the purpose of collection and processing (R38 and R39). The data subjects should be provided the duration their personal data will be stored (R55). Additional information must be provided to the data subjects if the collected personal data are required for other purposes (R37). General privacy-related information should be presented in a clear and simple, accessible language without technical terms as required in requirement (R26). Any updates of personal data processing must be informed to the recipients (i.e. processors or third parties) of those personal data (R50). The controllers must provide the contact details of responsible persons who control the processing (R20 and R22). The data subjects must be notified when they are likely to be in risk from personal data exposures (R15).}

\newtext{\subsection{User desirability}}
\newtext{This category consists of 9 requirements categorised into three sub-categories: consent, choice and preferences. The requirements in this category focus on ensuring that the processing of personal data is performed according to data subjects' consent and preferences. A number of requirements focus on the controllers being given authorities to process personal data (e.g. R8). It is also necessary to obtain consent for the processing based on those purposes (R35). The data subjects have options to allow the processing of their personal data for a certain specific purpose (R36).}

%%According to GDPR Art. 6, in addition to consent, there are other ways for a lawful processing if it is necessary for the performance of a contract, compliance with a legal obligation, protect vital interests, the performance of task carried out in the public interest and the purposes of the legitimate interests. While R35 covers the consent aspects, we have also R39 in our taxonomy addresses the other aspects. R39 requires that the controllers, if not obtaining the consent, must provide data subjects the purpose(s) of the processing of personal data, including those listed in GDPR Art. 6.

\newtext{\subsection{Data processing}}
%There are two requirements (R41 and R43) in this category, which change the traditional way of collecting and processing data. In the past, the data might be collected from data subjects as much as possible and kept in the system. They now require that the controllers are expected to collect and store only personal data that is required in the processing for the specific purpose(s). 

\newtext{This category addresses the processing of personal data from the controllers' side (16 requirements). The sub-categories in this category include collection, use, storage, erasure, transfer and record. The controllers are expected to collect and store only personal data that is required in the processing for the specific purpose(s) (R41 and R43). A set of requirements involves data erasure in systems. Requirement R53 addresses the case of removing personal data when the purpose for processing has expired. When the data subjects would like to have their personal data erased, the system shall provide this processing lawfully (R46 and R47). When the processing is complete and the personal data is no longer needed, the personal data should be removed from the system unless they are required by law/regulations (e.g. R51 and R52). In case that the personal data is unlawfully processed, the data must be removed from the systems (R7). In addition, personal data must be used only for the specified purpose(s) (R40). When requested by the data subjects, the controllers must transmit their personal data to another controller (R33). The data subject must be informed when their personal data needs to be transferred to a third country or an international organisation (R9). The controllers shall document the categories of personal data collected as it is important to know what personal data are stored in the systems (R70).}

\newtext{\subsection{Breach}}
This goal category focuses on providing and notifying important information related to personal data breaches to data subjects, relevant stakeholders and a supervisory authority (e.g. R15 and R71). Thus, it is important to implement a functionality that satisfies this compliance in the systems (e.g. R66). Requirement R67 imposes good practices of informing the related parties about the breaches. The controllers must be informed by processors about the breaches as well (R68). The controllers shall document the details of data breaches for verifying their compliance (R69).

\newtext{\subsection{Complaint/Request}}
\newtext{This privacy goal consists of 5 privacy requirements. It concerns complaint and request made by both data subjects and controllers. If the controllers refuse to take actions on the data subjects' requests about their individual rights, they have to provide a reason to the data subjects (R12). The data subjects must be able to lodge their complaints with a supervisory authority (R2). The controllers shall process personal data as requested (e.g. transmit personal data to another controller) (R33). The controllers should request additional information to confirm data subjects' identity when requests have been made (R31).}

\newtext{\subsection{Security}}
There are 13 requirements in our taxonomy covering the security practices in maintaining integrity, confidentiality and availability. The systems must allow only authorised people to access or process personal data (R56). The personal data should be protected with proper mechanisms (e.g. R60 and R63). The systems must restore the availability and access to personal data after incidents (R62). The interactions in the systems should neither identify nor observe the behaviour of the data subjects as well as reduce the linkability of the personal data collected (R64). Apart from the fundamental practices that the personal data should be protected, this is beneficial when the personal data is exposed. The data protection approaches such as anonymisation and pseudonymisation can help reduce the impact of privacy breaches. Most importantly, a set of requirements require the systems to implement mechanisms to ensure security and privacy compliance (e.g. R57, R58, R61 and R66). In addition, the implementation of mechanisms to assess the accuracy and quality of procedures should be considered (R49). 

In case that the personal data are processed across organisations/countries, the controllers must ensure local requirements and cross-border transfers (R65). Cross-border transfers are challenging for both controllers and processors. In cross-border transfer settings, it is required that the requirements at the destination should be equivalent to the ones at the source. The processors outside EU are sometimes not aware of those scenarios since they may not normally process personal data of EU citizens and residents. Therefore, the controllers are responsible for verifying requirements compliance before transferring personal data.

\newtext{All categories mentioned above can be considered as components in a software system. The privacy requirements are tasks in each component that the software engineers and relevant stakeholders should consider when designing and developing a system. Implementing those privacy requirements will fulfill the needs of stakeholders, which in this case is to satisfy the requirements stated in privacy and data protection regulations and frameworks.} \\
