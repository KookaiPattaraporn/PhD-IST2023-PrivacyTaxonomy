\section{Related work} \label{sec:related-work}

The protection of personal data and privacy of users have attracted significant attention in recent years. Data protection regulations and privacy frameworks have been established to guide the development of software and information systems. \emph{However, it is challenging for software engineers to translate legal statements stated in those regulations into specific privacy requirements for software systems \cite{Ayala-Rivera2018}, \cite{Aljeraisy2020}}. Data protection and privacy laws are independently designed and enforced for specific areas, which could range from a state (e.g. California), a country (e.g. Australia) or a region (e.g. Europe and Asia Pacific). Any organisations that meet the conditions of these laws must comply. However, the developers lack guidance and also experience difficulties in understanding and extracting such privacy requirements from those required laws \cite{Ayala-Rivera2018}, \cite{Aljeraisy2020}. Several studies have been calling for frameworks and methodologies to support software engineers in designing and developing privacy-aware software systems \cite{Gurses2011}, \cite{Senarath2018b}, \cite{Sheth2014}, \cite{Birnhack2014}. 

\citeauthor{Beckers2012} \cite{Beckers2012} proposed a conceptual framework to compare privacy requirements engineering approaches on requirements elicitation and notion representation. The approaches include LINDDUN, PriS and the framework for privacy-friendly system design approaches. The LINDDUN method elicits privacy requirements by modeling a system using a Data Flow Diagram (DFD) \cite{Deng2011}. The elements in the DFD are then mapped to the privacy threat categories to identify privacy requirements. \citeauthor{Kalloniatis2008} \cite{Kalloniatis2008} proposed a PriS method to elicit privacy requirements in the software design process. Privacy requirements in this study are modelled as a type of organisational goals that needs to be achieved in a specific application. Comparing with our study, LINDDUN and PriS methods do not elicit requirements from privacy and data protection regulations and frameworks, and they employed different requirements elicitation approaches.

Several existing work have aimed to identify privacy requirements and construct a privacy requirement taxonomy from privacy policies, regulations and standards. Ant\'{o}n et al. \cite{Anton2002}, \cite{Antn2004} used the GBRAM to develop the taxonomy of privacy requirements from the privacy policies of e-commerce and health care websites. The taxonomy was constructed by applying goal identification and refinement strategies to extract goals and requirements. We adapted this approach to construct the taxonomy presented in this paper. \citeauthor{Meis} \cite{Meis}, \cite{Meis2016} proposed a taxonomy of transparency requirements to support software engineers in identifying relevant requirements from a draft version of GDPR and ISO/IEC 29100. However, these studies emphasise on privacy goal transparency and intervenability in software development. \citeauthor{Gharib2017} \cite{Gharib2017} proposed an ontology that identifies key concepts for capturing privacy requirements. However, these concepts provide high-level dimensions rather than software requirements level. \citeauthor{Ayala-Rivera2018} \cite{Ayala-Rivera2018} proposed an approach to map GDPR data protection obligations with privacy controls derived from ISO/IEC standards. Those links help elicit the solution requirements that should be implemented in a software application. However, their study focused and validated only two articles in GDPR (i.e. Articles 5 and 25).

\newtext{The following work addressed the methods to elicit privacy requirements from software artifacts and patterns of GDPR requirements. \citeauthor{Colesky} introduced tactics which were used to link between privacy design strategies and privacy patterns \cite{Colesky}. The tactics can be considered as brief requirements which provide a guide to achieve privacy protection based on the privacy design strategies. This study associated those strategies, tactics and patterns to some GDPR entities and personal data processing examples. However, the tactics were defined in high-level and covered only one GDPR article.}

\newtext{\citeauthor{Ferreyra2020} proposed a method named PDP-ReqLite to elicit privacy and data protection requirements in systems and software projects \cite{Ferreyra2020}. It received Requirements DFD and Personal Information Diagram (PID) as inputs and generated meta-requirements in the form of pre-condition and post-condition predicates. The meta-requirements were patterns derived from translating the statements in GDPR directives and principles. Those meta-requirements were later combined into the respective GDPR categories. The study claimed that it ensured the full coverage of GDPR directives, however it only demonstrated the elicitation of undetectability requirements and did not clearly specify which GDPR articles were covered.}

\newtext{\citeauthor{Notario2015} developed a systematic methodology that combined risk-based and goal-oriented approaches to transform high-level privacy principles into operational privacy requirements \cite{Notario2015}. It identified relevant privacy principles based on organisational goals and/or regulatory frameworks, determined the required level of conformance for a system and determined the applicability of each requirement depending on other system and organisational constraints. However, the study only addressed the accountability principle in GDPR, and the method focused on software analysis and design processes.}

Several researchers proposed models and heuristics to represent and extract requirements from regulations. \citeauthor{Breaux2006} \cite{Breaux2006} proposed a process called Semantic Parameterisation to extract rights and obligations from the Privacy Rule from the U.S. Health Insurance Portability and Accountability Act (HIPAA). \citeauthor{Breaux2013} \cite{Breaux2013} developed a legal requirements specifications language (LRSL) to codify policy and law requirements from thirteen U.S. state data breach notification laws. This method also supports the traceability of regulatory requirements across multiple jurisdictions. However, these methods focused on specific schemes (i.e. HIPAA and breach notification).

Several work developed a tool support to automatically extract requirements from legal documents. \citeauthor{Zeni2015} \cite{Zeni2015} uses the semantic annotation (SA) technique to extract rights and obligations from HIPAA and Italian accessibility law for information technology instruments. The tool can capture the hierarchical structure and cross-references of legal documents and support the annotation of different languages other than English. \citeauthor{Sleimi2018} \cite{Sleimi2018} identified the metadata types of legal requirements from traffic laws and annotated the legal statements with the identified metadata types to generate NLP-based rules. Those rules were then implemented to automatically extract metadata types from the traffic laws of Luxembourg. Later, the same group of authors proposed a homonised set of legal requirements templates that systematically expresses legal requirements from multiple viewpoints \cite{Sleimi2020}. They also developed a tool support to automatically recommend those templates for the statements in the Luxembourg labour and health laws.

The following work discusses privacy requirements extraction from regulations for compliance checking in software systems. \citeauthor{Torre} \cite{Torre} developed a conceptual model using hypothesis coding to specify metadata types that exist in the statements of selected GDPR articles and created dependencies between those metadata types to ensure the proper completeness checking. They also employed Natural Language Processing (NLP) and Machine Learning (ML) techniques to automatically extract and classify the metadata in privacy policies from the fund domain. \citeauthor{Ghanavati2009} \cite{Ghanavati2009} developed a framework to analyse to what degree the organisation complies with laws. The study adopted a Goal-oriented Requirements Language (GRL) to model and analyse the relationship between organisational and legal requirements, and identify which organisational goals satisfy requirements in the law. This method was evaluated on Ontario's Personal Health Information Protection Act (PHIPA) only. 

Particularly focusing on the work related to GDPR, the European Commission has funded the GDPR cluster projects to help tackle the GDPR implementation challenges faced by organisations (e.g. \cite{BPR4GDPR}, \cite{DEFEND}, \cite{SMOOTH}, \cite{PDP4E}, \cite{PAPAYA}, \cite{POSEIDON} and \cite{Gharib2016a}). Those projects have developed both organisational and technical techniques to facilitate the implementation. They have addressed different challenges complying with GDPR in software development activities (e.g. planning, design, development, operation and deployment). They also provide solutions to the identified challenges. \citeauthor{EUcluster2020} \cite{EUcluster2020} has summarised the solutions proposed by some of these projects. 	


%\citeauthor{Beckers2012} \cite{Beckers2012} proposed a conceptual framework to compare privacy requirements engineering approaches on requirements elicitation and notion representation. The approaches include LINDDUN, PriS and the framework for privacy-friendly system design approaches. The LINDDUN method elicits privacy requirements from privacy threats \cite{Deng2011}. The method first developed nine categories of privacy threats: linkability, identifiability, non-repudiation, detectability, information disclosure, content unawareness and policy/consent non-compliance. Those threat categories are then mapped to the elements in data flow diagrams (DFDs) (i.e. entity, data flow, data store and process). The method further develops a catalogue of threat patterns (i.e. a threat tree) that represents common attacks concerned in each threat category and each element in the DFD. The patterns in the threat tree are mapped to the elements in DFD defined by misuse cases to identify privacy requirements. In the final step, the method recommends privacy-enhancing solutions that fulfill the identified requirements.

%\citeauthor{Kalloniatis2008} \cite{Kalloniatis2008} proposed a PriS method to elicit privacy requirements in the software design process. Privacy requirements in this study are modelled as a type of organisational goals that needs to be achieved in a specific application. The method first identifies privacy concerns in that specific application based on the eight pre-defined privacy goals (e.g. authentication and unlinkability). The second step is to analyse impact of privacy goals on relevant processes. This impact may introduce new goals or adjustment of existing goals. After having a set of privacy-related processes, the method models those processes based on the pre-defined privacy-process patterns of each privacy goal. The final step then suggests appropriate techniques for implementing those identified processes. Comparing with our study, LINDDUN and PriS methods do not elicit requirements from privacy and data protection regulations and frameworks, and they employed different requirements elicitation approaches. The taxonomies defined in their work are selected from a set of privacy properties identified in previous studies  \cite{Solove}, \cite{George2007}, \cite{Solove2008}, \cite{anon_terminology}, \cite{BritishStandardsInstitute2000}, whereas our taxonomy is formed after extracting privacy requirements from the regulations and standard.

%The requirements formulation process includes identifying verbs and related nouns, refining identified requirements and structuring the requirements into a taxonomy of transparency requirements. This approach was later used to derive intervenability requirements from the selected articles and principles in the draft of GDPR and ISO/IEC 29100 respectively \cite{Meis2016}. These intervenability requirements were then linked to the related transparency requirements proposed earlier in \cite{Meis}. However, these studies emphasise on privacy goal transparency and intervenability in software development. \citeauthor{Gharib2017} \cite{Gharib2017} proposed an ontology that identifies key concepts for capturing privacy requirements. However, these concepts provide high-level dimensions rather than software requirements level.

%Some existing work have aimed to identify privacy requirements and construct a privacy requirement taxonomy from privacy policies. \citeauthor{Anton2002} proposed a privacy goal taxonomy based on website privacy policies \cite{Anton2002}. The taxonomy was constructed by applying goal identification and refinement strategies based on the Goal-Based Requirements Analysis Method (GBRAM) \cite{Anton1996} to extract goals and requirements from 24 Internet privacy policies from e-commerce industries. This methodology was further used to develop the taxonomy of privacy requirements from 23 Internet health care Web sites privacy policies \cite{Antn2004}. They follow a goal mining process with heuristics to analyse and refine goals and scenarios from those privacy policies. We adapted this approach to construct the taxonomy presented in this paper.

%\citeauthor{Torre} \cite{Torre} proposed a different approach to check if an organisation's  privacy  policies  comply  with  GDPR. They first developed a conceptual model using hypothesis coding to specify metadata types that exist in the statements of selected GDPR articles. They created the dependencies between the metadata types to ensure the proper completeness checking. They then employed Natural Language Processing and Machine Learning techniques to automatically classify the information content in privacy policies from the fund domain and determine the extent to which the content satisfies a certain completeness criterion. However, the metadata types and their dependencies in the model are limited as several key articles in the GDPR were not considered (e.g. data subjects' rights and security of processing). \newtext{In addition, additional criteria for completeness checking should be considered since the presence or absence of those metadata may not be sufficient to determine the completeness of statements in privacy policies against GDPR.}


%The following work discusses privacy requirements extraction from regulations and privacy policies for compliance checking in software systems. \citeauthor{Muller2019} \cite{Muller2019} developed a dataset consisting of statements in organisations' privacy policies. Those statements were annotated with only five privacy requirements extracted from GDPR. This dataset assisted the development of automated tools for checking GDPR compliance of an organisation's privacy policies. \citeauthor{Torre} \cite{Torre} proposed a different approach to check if an organisation's  privacy  policies  comply  with  GDPR. They first developed a conceptual model using hypothesis coding to specify metadata types that exist in the statements of selected GDPR articles. They created the dependencies between the metadata types to ensure the proper completeness checking. They then employed Natural Language Processing and Machine Learning techniques to automatically classify the information content in privacy policies from the fund domain and determine the extent to which the content satisfies a certain completeness criterion. However, the metadata types and their dependencies in the model are limited as several key articles in the GDPR were not considered (e.g. data subjects' rights and security of processing). In addition, additional criteria for completeness checking should be considered since the presence or absence of those metadata may not be sufficient to determine the completeness of statements in privacy policies against GDPR.

%\citeauthor{Torre2019} \cite{Torre2019} explored the use of models to assist GDPR compliance checking. They developed a UML class diagram representation of GDPR. This UML model is enriched with invariants expressed in the Object Constraint Language to captures GDPR rules. \citeauthor{Pandit2019} \cite{Pandit2019} developed an ontology of generic concepts and relationships of the components identified in GDPR. This ontology is used for completeness and compliance checking in privacy policies. The taxonomy addresses only six articles in the GDPR.

%\citeauthor{Pandit2019} \cite{Pandit2019} developed the Data Privacy Vocabulary (DPV) which is an ontology of generic concepts and relationships of the components identified in GDPR. This ontology is used for completeness and compliance checking in privacy policies, consent receipts and records of personal data handling. However, their taxonomy does not cover any specific software requirements. In addition, the taxonomy addresses only six articles in the GDPR, while our work covers more articles in the GDPR and other well known data protection regulations and privacy frameworks.

\begin{comment}

Original GDPR-funded projects

Particularly focusing on the work related to GDPR, the European Commission has funded the GDPR cluster projects to help tackle the GDPR implementation challenges faced by organisations \cite{EUcluster2020}. Those projects have developed both organisational and technical techniques to facilitate the implementation. They have addressed different challenges complying with GDPR in software development activities (e.g. planning, design, development, operation and deployment). They also provide solutions to the identified challenges. \citeauthor{EUcluster2020} \cite{EUcluster2020} has summarised the solutions proposed by these projects. 	

The Business Process Re-engineering and functional toolkit for GDPR compliance project\footnote{https://www.bpr4gdpr.eu/} (BPR4GDPR) provides an approach and a toolkit to support end-to-end GDPR-complaint business processes, particularly for small and medium-sized enterprises (SMEs) \cite{BPR4GDPR}. The deliverables include the policy-based access and usage control framework, specification of workflow models and tools for cryptography, access management and enforcement of data subjects' rights \cite{EUcluster2020}.

As there is neither specific methods, techniques or tools to evaluate the GDPR readiness level in organisations nor privacy governance guideline available, the Data Privacy Governance for Supporting GDPR project\footnote{https://www.defendproject.eu/} (DEFeND) introduces a platform to assist in complying with the GDPR \cite{DEFEND}. The platform supports organisations in designing and developing tailored solution that covers different aspects of GDPR. It also provides methods and techniques to handle personal data and consent management as well as data protection mechanisms in software systems.

SMOOTH project\footnote{https://smoothplatform.eu/about-smooth-project/} aims to provide support for micro enterprises to adopt GDPR as a cloud-based platform. The solution helps assess the compliance of the enterprises and identify the basic requirements that need to be satisfied to comply with GDPR \cite{SMOOTH}. After the enterprise provides the information related to its current data protection management through the questionnaire in the platform, the platform will generate a compliance report with appropriate guidance to resolve non-compliance in the enterprise.

The Privacy and Data Protection 4 Engineering (PDP4E) project\footnote{https://www.pdp4e-project.eu/} develops a set of systematic methods and tools to address the following disciplines in software development life cycle: risks assessment, requirements engineering, model-driven design and systems assurance \cite{PDP4E}. Particularly focusing on requirements engineering, the project has introduced PDP-ReqLite, an approach to elicit privacy and data protection meta-requirements using a lightweight version of problem frames \cite{Ferreyra2020}.

The PlAtform for PrivAcY preserving data analytics (PAPAYA) project \footnote{https://www.papaya-project.eu/} develops a platform that addresses privacy preservation in data analytics. Finally, the Protection and control of Secured Information by means of a privacy enhanced Dashboard (PoSeID-on) project \footnote{https://www.poseidon-h2020.eu/} provides a solution allowing end users to manage their personal data and safeguarding the rights of data subjects \cite{POSEIDON}. The project also ensures GDPR-compliant data management and processing for organisations. The project develops a dashboard presenting the summary of their personal data (e.g. the personal data that is allowed to be shared and history of personal data transactions). The end users will be notified when data processors require the permissions to use their personal data. The data processors can also check what personal data is shared to them.

\end{comment}