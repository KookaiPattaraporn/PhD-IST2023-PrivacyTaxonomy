\section{Introduction} \label{sec:intro}

%Software applications have become an integral part of our society and people's lives. Digital footprints are collected as people are online browsing the Internet or using various software applications such as for social networking, working, studying and leisure activities. Physical footprints are also collected through software systems such as surveillance cameras, face recognition apps, IoT sensors and GPS devices even when people are ``offline'' doing their normal life activities. Zettabytes of those data are collected and processed for various purposes, including extracting and utilising personal data, and forming behavioural profiles of individuals. This poses serious threats to our privacy and the protection of our personal sphere of life -- the cornerstone of human rights and values.

Software applications have become an integral part of our society. Digital trails are collected as people are browsing the Internet or using various software applications such as for social networking, working, studying and leisure activities. Physical trails are also collected through software systems such as surveillance cameras, face recognition apps, IoT sensors and GPS devices even when people are ``offline'' doing their normal life activities. Zettabytes of those data are collected and processed for various purposes \cite{Statista2021a}, including extracting and using personal data, and forming behavioural profiles of individuals. This poses serious threats to our privacy and the protection of our personal sphere of life -- the cornerstone of human rights and values.

Organisations have been collecting personal data of their customers for various business purposes. Cyberattacks often target at obtaining this data. CSO Online reported the fourteen biggest data breaches of the 21st century that affected 3.5 billion people \cite{Swinhoe2020}. The cases occurred with the world's top software applications, for examples, Adobe, Canva, eBay, LinkedIn and Yahoo.

The recent advent of privacy legislations, policies and standards (e.g. the European's General Data Protection Regulation \cite{OfficeJournaloftheEuropeanUnion;2016} or the ISO/IEC standard for privacy framework in information technology \cite{ISO/IEC2011}) aims to mitigate those threats of privacy invasion. A range of frameworks (e.g. privacy by design) and privacy engineering methodologies have also emerged to help design and develop software systems that provide acceptable levels of privacy and meet privacy regulations \cite{Ayala-Rivera2018}, \cite{Deng2011}, \cite{Aljeraisy2020}. However, those methodologies provide only high-level principles and guidelines, leaving a big gap for software engineers to fill in designing and implementing privacy-aware software systems \cite{Gurses2011}. Software engineers often face challenges when navigating through those regulations and policies to understand and implement them in software systems \cite{Ayala-Rivera2018}, \cite{Aljeraisy2020}, \cite{Senarath2018b}.

%In 1995, \citeauthor{Cavoukian2009} initially developed a reference framework called \textit{Privacy by Design (PbD)} aiming to emphasise the importance of privacy in the engineering processes of information systems \cite{Cavoukian2009}. The framework describes seven principles that provide conceptual characteristics of privacy elements related to the processing of personal data. However, this framework is often criticised as being too broad and vague for actual implementation \cite{Rest2014}. Recently, \textit{privacy engineering} is an emerging field which addresses privacy concerns in the development of sociotechnical systems \cite{Gurses2016}. Unfortunately, the integration of privacy solutions into practices has faced several challenges and limitations due to the design of organisational processes \cite{Spiekermann2012}, \cite{Mikkelsen2019}, legacy IT systems \cite{Capgemini2019}, implementation cost \cite{Capgemini2019} and development practices \cite{Bednar2019}, \cite{Hadar2018}. 

Hence, there is an emerging need to translate complex privacy concerns set out in regulations and standards into requirements that are to be implemented in software applications. Such privacy requirements need to be refined into a level that emphasises the functionalities in software systems and can be later used to map with issue reports. Previous work has involved extracting privacy requirements, but they are specific to a certain application domain such as e-commerce applications (e.g. \cite{Anton2002}) or healthcare websites (e.g. \cite{Antn2004}). More recent studies (e.g. \cite{Anthonysamy2017} and \cite{Guarda2009}) revealed an urgent need for a reference taxonomy of privacy requirements that are based on well-established regulations and standards such as GDPR and ISO/IEC 29100.

This taxonomy would be useful to understand how privacy requirements are explicitly addressed in software projects, and also serve as a basis for developing future privacy-aware software applications. Most of today's software projects follow an agile, issue-driven style in which feature requests, functionality implementations and all other project tasks are usually recorded as issues (e.g. JIRA issues\footnote{https://www.atlassian.com/software/jira}) \cite{Choetkiertikul}. In those projects, issue reports essentially contain important, albeit implicit, information about the requirements of a software, in the form of either new requirements (i.e. feature requests), change requests for existing requirements (i.e. improvements) or reporting requirements not being properly met (i.e. bugs) \cite{Choetkiertikul}, \cite{MoodleTracker}, \cite{MoodleFeature}. A software project consists of past issues that have been closed, ongoing issues that the team are working on, and new issues that have just been created. Through a study of those issues in the project, we can understand how the software team has implemented privacy requirements recorded in the issue tracking system (ITS) in order to address relevant privacy needs and concerns of stakeholders. This paper provides the following contributions:

\begin{enumerate}[leftmargin=0.5cm]
  \item We developed a comprehensive taxonomy of privacy requirements for software systems by extracting and refining requirements from the widely-adopted GDPR and ISO/IEC 29100 privacy framework as well as the newly developed Thailand PDPA and the region-specific APEC privacy framework. We followed a grounded theory process adapted from the Goal-Based Requirements Analysis Method (GBRAM) \cite{Antn2004} to develop this taxonomy. The taxonomy consists of 7 categories and 71 privacy requirement types. To the best of our knowledge, this taxonomy is well-grounded in standardised privacy regulations and frameworks as it covers more regulations and frameworks and the number of articles compared to the existing work (e.g. \cite{Meis}).  
  %this is the \emph{first} taxonomy of privacy requirements that are well grounded in standardised privacy regulations and frameworks.

  \item We mined the issue reports of two large-scale software projects, Chrome and Moodle (each has tens of thousands of issues) to extract 1,374 privacy-related issues. We classified all of those issues into the privacy requirements of our taxonomy. The classification was performed by multiple coders through multiple rounds of training sessions, inter-rater reliability assessments and disagreement resolution sessions. This resulted in a reliable dataset for the research community to perform future research in this timely, important topic of software engineering such as automated classification of privacy issue reports.

 \item We studied how the privacy requirements in our taxonomy were addressed in Chrome and Moodle issue reports. We found \newtext{2,432} occurrences of the privacy requirements in our taxonomy were covered in those datasets (see Section \ref{sec:results} for more details), most of which related to the user participation category. In addition, we found that allowing the erasure of personal data is a top concern reported in the Chrome and Moodle issue reports, while none of the privacy requirements related to the management perspective of data controllers was recorded in the issue tracking system of both projects. We also discovered that privacy and non-privacy issues were treated differently in terms of resolution time and developers' engagement in both projects.

 % \item A prototype tool which can automatically classify a privacy issue into appropriate privacy requirement categories. Our recommendation model using TF-IDF and Random Forests achieves 0.769 MAP and 0.859 MRR. It improves over 36\% of MAP and 19\% MRR compared to two baseline methods (averaging across two projects).

\end{enumerate}

A full replication package containing all the artifacts and datasets produced by our studies are made publicly available at \cite{reppkg-pridp}. The remainder of this paper is structured as follows. Section \ref{sec:related-work} provides related existing work on privacy requirements engineering. \newtext{A theoretical background of the privacy requirements taxonomy is discussed in Section \ref{sec:theoretical-background}. The methodology used to build the taxonomy is presented Section \ref{sec:taxonomy-development}. Section \ref{sec:taxonomy} describes the privacy requirements taxonomy and its categories in detail. Section \ref{sec:mining} presents our study of how Chrome and Moodle issue reports address the privacy requirements in our taxonomy. The findings and insights generated from the study are discussed in Section \ref{sec:results}. Section \ref{sec:threats} discusses the threats to validity. Finally, we conclude and discuss future work in Section \ref{sec:conclusion}.}