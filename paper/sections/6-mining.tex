\section{\newtext{Identifying privacy requirements in issue reports}} \label{sec:mining}

Most of today's software projects follow an agile style in which the software development is driven by resolving issues in the backlog. In those projects, issue reports contain information about requirements of a software that are recorded in multiple forms in an issue tracking system \cite{Choetkiertikul}: new requirements (such as user stories or new feature request issues), modification of existing requirements (such as improvement request issues) or reporting requirements not being properly met (i.e. bug report issues). Large projects may have thousands of issues which provide fairly comprehensive source of requirements about the projects and the associated software systems. Thus, we have performed a study on the issue reports of software projects to understand how the selected projects address the privacy requirements in our taxonomy. As issue reports can be in multiple forms, we note that the taxonomy can be applied to software requirements and user stories. In this section, we first describe the process steps that we have followed to carry out our study, assess classification results, resolve disagreements, and discuss the outcomes.

\begin{figure}[ht]
	\centering
	\includegraphics[width=.9\linewidth]{"Figures/Mining-issue-reports"}
	\caption{An overview process of identifying and classifying privacy requirements in issue reports}
	\label{fig:Mining issue reports}
\end{figure}

\subsection{Issue reports collection}

We apply a number of criteria to select software projects for this study as follows: (i) are open-source projects, (ii) serve a large number of users, (iii) are related to privacy and (iv) have accessible issue tracking systems. There were a number of projects which satisfy these criteria. Among them, we selected Chrome and Moodle due to their large scale size, popularity and representativeness\footnote{As discussed later in the paper, performing thorough analysis as we have done on these two projects alone required significant effort (278 person-hours). Hence, we scoped our study in these two projects so that we can publish this dataset timely, enabling the community to initiate research on this important topic, and subsequently extend it with additional projects and data.}. Google Chrome is one of the most widely-used web browsers, which was developed under the Chromium Projects \cite{Projects}. As a web browser, Chrome stores personal data of users, e.g. username, password, address, credit card information, searching behaviour, history of visited sites and user location. Moodle is a well-known open-source learning platform \cite{Moodle} with over 100 million users worldwide. Moodle aims to comply with GDPR \cite{Moodle2019}.

%The process steps for mining and classifying issue reports are as follows.
Figure \ref{fig:Mining issue reports} shows the process steps that we have followed in our study. We first identified privacy-related issues from all the issue reports we collected from Chrome and Moodle projects. To do so, we identified issues that were explicitly tagged as privacy in the ``component'' field of both projects and their status was either assigned, fixed and verified (to ensure that they are valid issues). This is to ensure that the issue reports we collected were explicitly tagged as ``privacy'' by Chrome and Moodle contributors. This process initially gave us 1,080 privacy-related issues from Chrome and 524 from Moodle. We then manually examined those issues and filtered out those that have limited information (e.g. the description that does not explain the issue in detail or contain only source code) to enable us to perform the classification. For example, issue ID 953622\footnote{https://bugs.chromium.org/p/chromium/issues/detail?id=953622} states that ``Null-dereference READ in bool base::ContainsKey<std::\_\_Cr::map<std::\_\_Cr::basic\_string<char, std::\_\_Cr::c''. This issue description is very brief and does not contain any explanation describing what the issue is about, what personal data is concerned or which function is affected in the issue. We thus exclude the issue from our study. Finally, our data contains 896 issues from Chrome and 478 issues from Moodle.

In Chrome dataset, the collected issue reports were created between January 2009 and March 2020. There are five issue types reported in our dataset: bug, bug-regression, bug-security, feature and task. Each issue report has seven contributors on average; the contributors include reporters, owners and relevant collaborators. Issue reports in Moodle dataset span for two years, 2018 - 2019. The issue types in Moodle include bug, epic, improvement, new feature, task, sub-task and functional test. On average, five participants are involved in each report including reporters, assignees, testers and commenters. The descriptive statistics of the issue reports can be seen in Table \ref{tab:issue-stats}\footnote{The number of issue reports counted by issue type in both projects is provided in the supplementary material.}.

\begin{table*}[h]
	\centering
	\caption{Descriptive statistics of the number of contributors, resolution time and number of comments of the issue reports in our datasets.}
	\label{tab:issue-stats}
	\resizebox{6.5in}{!}{%
		\begin{tabular}{@{}llllllllllllllll@{}}
			\toprule
			\multirow{2}{*}{\textbf{Project}} & \multicolumn{5}{c}{\textbf{\#Contributors}} & \multicolumn{5}{c}{\textbf{Resolution Time (days)}} & \multicolumn{5}{c}{\textbf{\#Comments}} \\ \cmidrule(l){2-16}
			& \textbf{min} & \textbf{max} & \textbf{mean} & \textbf{median} & \textbf{mode} & \textbf{min} & \textbf{max}  & \textbf{mean} & \textbf{median} & \textbf{mode} & \textbf{min} & \textbf{max} & \textbf{mean} & \textbf{median} & \textbf{mode} \\ \midrule
			Google Chrome & 1   & 32  & 5    & 4      & 2    & 1   & 3,635 & 315  & 65     & 1    & 0   & 311 & 16   & 12     & 12    \\
			Moodle        & 1   & 14  & 4    & 5      & 5    & 1   & 852   & 37   & 13     & 1    & 0   & 112 & 11   & 9     & 1    \\ \bottomrule
		\end{tabular}%
	}
	{\parbox{16.5cm}{\footnotesize \#Contributors: number of contributors, \#Comments: number of comments, min: minimum of contributors/resolution time/comments, max: maximum of contributors/resolution time/comments, mean: mean of contributors/resolution time/comments, median: median of contributors/resolution time/comments, mode: mode of contributors/resolution time/comments.}}
\end{table*}

\begin{comment}

\begin{table}[h]
	\centering
	\caption{Number of issue reports in Chrome and Moodle projects counted by issue type.}
	\label{tab:issue-types}
	\resizebox{3.25in}{!}{%
		\begin{tabular}{@{}llll@{}}
			\toprule
			\textbf{Chrome} & \textbf{\#issues} & \textbf{Moodle} & \textbf{\#issues} \\ \midrule
			Bug             & 620               & Bug             & 223               \\
			Bug-regression  & 59                & Epic            & 3                 \\
			Bug-security    & 36                & Improvement     & 101               \\
			Feature         & 132               & New Feature     & 75                \\
			Task            & 5                 & Task            & 26                \\
			Unspecified     & 44                & Sub-task        & 37                \\
			&               	& 					Functional Test & 13 \\ \bottomrule
		\end{tabular}%
	}
\end{table}

\end{comment}


\subsection{Issue reports classification}

In this phase, we went through each issue report in the dataset to classify it into the privacy requirements in our taxonomy. This phase consists of three steps: (i) identifying concerned personal data described in the issue report, (ii) identifying functions/properties reported in the issue, (iii) mapping the issue to one or more privacy requirements. 

Regarding the classification, each coder was initially provided with an online form containing the title and description of the assigned issue reports and 71 columns representing each requirement. The coders analysed each issue following the classification steps described above (i.e. steps (i) - (iii)). The coders carefully consider every scenario mentioned in the issue reports. Once the coders have identified related information about personal data and function(s) concerned, they considered the relevant requirements. The coders determined the requirement(s) that matches with information analysed above. The coders then updated the value in the columns of chosen requirement(s) in the given form. Finally, all the coders delivered their result file containing the issue reports and their privacy requirement labels for reliability assessment process. To ensure that the classification process is reliable, two coders were assigned to classify an issue report. The reliability assessment and disagreement resolution processes are described in detail in Section \ref{subsec:mining-reliability}.

The following example demonstrates the issue reports classification in our datasets. Issue 123403\footnote{https://bugs.chromium.org/p/chromium/issues/detail?id=123403} in Chrome reports that \textit{``Regression: Can't delete individual cookies''}. The personal data affected here is individual cookies, and the function reported is erasing or deleting (individual cookies). Thus, we classify this issue into the requirement \textit{\textbf{ALLOW} the data subjects to erase their personal data (R44)} in our taxonomy (see Table \ref{tab:sample-requirements}). The users should be able to select the cookies that they want to delete.

In another example, issue 495226\footnote{https://bugs.chromium.org/p/chromium/issues/detail?id=495226} in Chrome requests that the \textit{``Change Sign-in confirmation screen''} should be changed. The description of this issue requires that the system should inform the reasons for user account data collection and how this data will be further processed before obtaining this data in the sign-in process. Since this issue requires that the user should be informed of the purpose of collection and processing, the issue can be classified into requirements R38 and R39 (see Table \ref{tab:sample-requirements}). Both of these requirements belong to the notice privacy goal. This example shows that an issue can be classified into more than one privacy requirements.

Issue 831572\footnote{https://bugs.chromium.org/p/chromium/issues/detail?id=831572} in Chrome requires: \textit{``Provide adequate disclosure for (potentially intrusive) policy configuration''}. Further investigation into the issue's description revealed that the disclosure of policy configuration includes: letting the users know that they are managed, and providing indication when user data may be intercepted and when user actions are logged locally. These involve the following functions: (i) the users should be informed of the purpose of processing so that they know they are managed; (ii) the enterprise may intercept the users' data, thus the users should know whom their personal data might be sent to; and (iii) the history of user logging shall be recorded to acquire logging data. Hence, this can be classified into three requirements R39, R27 and R13 (see Table \ref{tab:sample-requirements}). This example demonstrates that one issue relates to several requirements across different privacy goal categories: notice and data processing.

The following example demonstrates the issue that concerns multiple functions which more than one privacy goal is addressed in Moodle. Issue MDL-62904\footnote{https://tracker.moodle.org/browse/MDL-62904} in Moodle reports that \textit{``users can't find where to request account deletion''}. The issue was described that the system does not provide a function for users to request for deleting their account in the user interface. Hence, this issue addresses requirements R30 in the notice category and R44 in the user participation category in the taxonomy.

Although the mapping is relatively straightforward in most of the issues, some presents challenges. For instance, a set of issues in Moodle refer to the implementation of the core\_privacy plugins (e.g. MDL-61877\footnote{https://tracker.moodle.org/browse/MDL-61877}). However, the information given in the description of those issues is inadequate to identify which privacy requirements are related to. Therefore, we needed to seek for additional information about core\_privacy plugins in Moodle development documentation \cite{Moodle2019}. In addition, the issues in Chrome and Moodle projects have specific function names or technical terms used by their developers. Hence, extra effort was required to understand those issues and classify them. Once the coders acquired the additional information from software documentation, they discussed potential privacy issues/functionalities raised in those issue reports. For example, the documentation of core\_privacy plugins explains six functionalities that the plugins should provide. The coders then discussed which privacy requirements in the taxonomy involved with those functionalities, and classified that issue according to the identified privacy requirements.

\subsection{Reliability analysis} \label{subsec:mining-reliability}

The classification process has been performed by three coders, who also involved in the taxonomy development process. All the coders have independently followed the above process to classify issue reports into our taxonomy of privacy requirements. The first coder was responsible for classifying all 1,374 privacy issue reports. The classification process is however labour intensive. It took the first coder approximately 138 person-hours to classify all 1,374 issue reports (6 minutes per issue report in average). The second and third coders spent approximately 70 hours per person to classify 687 issue reports. In total, the process took four months to complete, so the coders had time to manage their workloads efficiently. The coders usually divided the assigned issue reports into smaller sets (i.e. about 50 issue reports) to work on for each round. For the purpose of reliability assessment, the second and third coder each was assigned to classify a different half of the issues in each project. This setting aimed to ensure that each issue report is classified by at least two coders.

An issue report can be classified into multiple privacy requirements (i.e. a multi-labelling problem). Hence, we employ Krippendorff's alpha coefficient \cite{Krippendorff2011}, \cite{Artstein2008} with MASI (Measurement Agreement on Set-valued Items) distance to measure agreement between coders with multi-label annotations \cite{Ravenscroft2016}, \cite{Passonneau2006}. The MASI distance measures difference between the sets of labels (i.e. privacy requirements) provided by two coders for a given issue. The Krippendorff's alpha values between the first and second coders are 0.509 for Chrome and 0.448 for Moodle. The agreement values between the first and third coder are 0.482 and 0.468 for Chrome and Moodle respectively. A disagreement resolution step was conducted to resolve the classification deviations between the three coders\footnote{Krippendorff’s alpha values indicate the degree of (dis-)agreement between coders. In our case, disagreements were often due to the ambiguity in the issue reports, requiring us to perform a resolution step. Doing these is to increase the reliability of our dataset.}.

The low Krippendorff's alpha values from the initial classification were \emph{not} due to the privacy requirements in our taxonomy. Rather, it was mainly because of the limited information provided in the description of a number of issue reports, forcing the coders to make their own assumptions about the nature of those issues. If an issue report is clearly described, the coders classified it into the same requirements. 53.01\% of the Chrome issues and 46.23\% in Moodle received this total agreement between all the three coders. We have addressed this problem by conducting a disagreement resolution step where all the coders met and discussed to resolve the disagreements. 

%One major reason leads to disagreements was the coders' interpretation of issue reports, especially in the cases where the issues have limited and/or vague description. If an issue report is clearly described, the coders classified it into the same requirements. 53.01\% of the Chrome issues and 46.23\% in Moodle received this total agreement between all the three coders. The specificity of privacy requirements was also another reason for disagreements. Some requirements in the taxonomy are detailed with specific conditions (e.g. six different ways to remove personal data). Some issue reports did not cover those specific conditions, thus the coders needed to interpret it from the given context of the issue and chose the most suitable requirement.

\newtext{\textbf{Disagreement resolution}:} \newtext{We conducted several meeting sessions between the coders to resolve disagreements in a sample dataset. The sample dataset contained the issue reports that were classified into different privacy requirements by both coders. Specifically, both coders did not classify those issue reports into at least one same requirement. There were 343 and 161 issue reports in Google Chrome and Moodle resolved in the meeting sessions between the coders who were responsible for the classification.} Several meetings were conducted because of the time difference and availability among the coders. Each meeting took 3 hours on average as we revisited, discussed and reclassified the issues with disagreements issue by issue. The same coders continue working on the disagreed issues in the same set of issues they had annotated in the previous step. During these meetings, the coders examined the issues thoroughly (not just only their description, but also other documentation related to the issues), discussed to develop a mutual understanding of the issue, and then reclassified the issue together. For each issue in the list, each coder explained the justification for their classification of the issue. The coders then resolved the disagreements on that issue in two ways. After the discussion, if the coders agreed with the other coder's classification, the labels were combined (i.e. the issue were classified into multiple requirements). If the coders did not reach an agreement, they went through the issue's description together to discuss and develop a mutual understanding of the issue. They then reclassified the issue together. Hence, the final classification has maximum agreement among the coders, thus ensuring the reliability of our dataset. \newtext{Once all disagreements had been resolved for the sample set, the first coder adjusted the classification of the issues which were not included in the sample set to finalise the datasets. This included 78 Chrome and 96 Moodle issue reports respectively.}

%We conducted several meeting sessions between the coders to resolve disagreements. Several meetings were conducted because of the time difference and availability among the coders. Each meeting took 3 hours on average as we revisited, discussed and reclassified the issues with disagreements issue by issue. The same coders continue working on the disagreed issues in the same set of issues they had annotated in the previous step. During these meetings, the coders examined the issues thoroughly (not just only their description, but also other documentation related to the issues), discussed to develop a mutual understanding of the issue, and then reclassified the issue together. For each issue in the list, each coder explained the justification for their classification of the issue. The coders then resolved the disagreements on that issue in two ways. After the discussion, if the coders agreed with the other coder's classification, the labels were combined (i.e. the issue were classified into multiple requirements). If the coders did not reach an agreement, they went through the issue's description together to discuss and develop a mutual understanding of the issue. They then reclassified the issue together. Hence, the final classification has maximum agreement among the coders, thus ensuring the reliability of our dataset. Once all disagreements have been resolved for the sample set, the first coder adjusted the classification of the issues which are not included in the sample set to finalise the dataset. 

%We conducted several meeting sessions between the coders to resolve disagreements in a sample dataset. The sample dataset contains the issue reports that were classified into different privacy requirements by both coders. Specifically, both coders did not classify those issue reports into at least one same requirement. There were 343 and 161 issue reports in Google Chrome and Moodle resolved in the meeting sessions between the coders who were responsible for the classification. \\

%Once all disagreements have been resolved for the sample set, the first coder adjusted the classification of the issues which are not included in the sample set to finalise the dataset. This included 78 Google Chrome and 96 Moodle issue reports respectively.

The following example illustrates how conflicts between coders were resolved. Issue 527346\footnote{https://bugs.chromium.org/p/chromium/issues/detail?id=527346} in Chrome requests that the users should know when they are managed. The description of the issue requires the system to show information to users when they are managed and the information should be easily seen by users. This issue was classified to R26 by one coder and R30 by the other coder. Although both R26 and R30 involve providing information to users, R26 focuses on the information relating to the policies, procedures, practices and logic of the processing of personal data, while R30 focuses on representing information relating to the processing of personal data with standardised icons in the user interface. After the coders revisited the issue and discussed the description in detail, the coders agreed that the information should be shown in the tray bubble which is a part of the user interface. The coders therefore reclassified this issue into R30.

%%Our work helps establish the traceability between issue reports and privacy concerns expressed in the regulations, standards and frameworks. These suggest what should be done to resolve the concerns raised in issue reports and meet the associated privacy requirements. In addition, this traceability facilitates compliance checking with respect to some specific regulations, standards and/or frameworks. The following examples demonstrate the privacy requirements traceability in issue reports. Referring to issue 123403 in Section 4.2, this bug-regression issue report concerns requirement R44 in the taxonomy. Requirement R44 was derived from GDPR, ISO/IEC 29100, Thailand PDPA and APEC privacy framework. To resolve this issue and comply with all four regulations, standards and frameworks, the system must provide a functionality for the controller to allow the data subjects to erase their personal data. In another example, referring to issue 123403 in Section 4.2, it addresses two privacy requirements, R30 and R44. Requirement R30 was derived from GDPR and APEC privacy framework. To resolve this issue and to comply with GDPR and APEC privacy framework, the system must also provide a functionality for the controller to provide the data subjects the information relating to the processing of personal data with standardised icons.

%%Our approach can be also applied to establish privacy requirements traceability in other software artifacts such as design models, source code and test cases. In general, our approach can be used a reference framework in developing privacy-aware software systems. For example, when a software engineer develops a system functionality which collects personal data, the software engineer may consult the taxonomy to identify the requirements related to the information required to provide to the data subjects before collecting their personal data such as concerned personal data (R42), purposes of collection (R38), purposes of processing (R39) and period/criteria used to store personal data (R55). The privacy requirements were derived from the regulations, standards and frameworks, hence meeting those requirements forms a basis for privacy compliance.

%%Although issue reports are a good source of information for software requirements, we acknowledge that they are not the only source. Software requirements can be in other forms such as requirements specifications or other documents (such as Confluence\footnote{https://www.atlassian.com/software/confluence/features} pages). The privacy requirements in the taxonomy can also be mapped to a list of privacy-related items concerned in a range of checks in organisations such as Data Protection Impact Assessment (DPIA). For example, DPIA requires a system to obtain consent from data subjects before processing their personal data. Requirement R35 in the taxonomy addresses this concern. However, the scope of this work focuses on privacy requirements in issue reports, and we plan to explore privacy requirements in other sources in our future work.