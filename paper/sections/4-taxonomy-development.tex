\section{\newtext{Privacy Requirements Taxonomy Development}} \label{sec:taxonomy-development}

This section discusses the methodology that we have followed to develop this taxonomy  (see Figure \ref{fig:taxonomy development}). We followed a content analysis process adapted from the GBRAM \cite{Antn2004}, which is based on Grounded Theory, to develop a taxonomy of privacy requirements. GBRAM is a systematic method used to identify, refine and organise goals into software requirements. This process was applied to analyse goals from natural language texts in privacy policies, and convert them into software requirements. The method has been successfully applied to the analysis of e-commerce applications \cite{Anton1998} and health care privacy policies \cite{Antn2004}. The method consists of three main activities: goal identification, goal classification and goal refinement. Goal identification derives goals from specifications in selected sources. Each identified goal is then classified into one of the pre-defined categories in the goal classification. Finally, the goal refinement removes synonymous and redundant goals, resolves inconsistencies among the goals and operationalises the goals into requirements specification. 

We had multiple researchers (co-authors of the paper, hereby referred to as the coders) follow this process to develop the taxonomy independently, and used the inter-rater reliability (IRR) assessment to validate the agreements and resolve disagreements. Those coders were given instructions and trained at the start of the process. The process consists of the following steps:

\begin{itemize}[leftmargin=*, noitemsep]
	\item \textbf{\newtext{Step 1} - privacy requirements identification:} extract requirements from written statements in the studied privacy regulations and frameworks, and structure them into a pattern (action verb, objects and object complement).
	
	\item \textbf{\newtext{Step 2} - privacy requirements refinement:} remove duplicate requirements and manage inconsistent requirements. Since the inputs were written in descriptive statements and from different sources, requirements can be redundant or inconsistent.
	
	\item \textbf{\newtext{Step 3} - privacy requirements classification:} classify requirements into categories based on a set of privacy goals. The privacy goals can be considered as a group of functionalities that the software systems are expected to provide.
\end{itemize}

\begin{figure}
	\centering
	\includegraphics[width=1\linewidth]{"Figures/Taxonomy-development"}
	\caption{\newtext{An overview process of privacy requirements taxonomy development}}
	\label{fig:taxonomy development}
\end{figure}

The details of each step are described as follows.
\newtext{\subsection{Privacy requirements identification}} \label{subsec:req-identification}

This step aims to identify privacy requirements from the narrative statements in GDPR, ISO/IEC 29100 privacy framework, Thailand PDPA and APEC privacy framework. We first created a range of questions to identify goals from each statement of the studied privacy regulations and frameworks. As part of our research, we have carefully manually gone through all 99 articles in the GDPR. Although GDPR and Thailand PDPA govern broader regulatory aspects comparing to ISO/IEC 29100 and APEC privacy frameworks such as requirements in roles assignment (e.g. data protection officer and supervisory authority), managing juridical remedies and noticing penalties, those are not software requirements. Thus, they are out of the scope of our study. 

We then applied several filters to select the articles that address software requirements in GDPR and include them in our study. We selected 19 articles that address the rights of individuals and cover the key principles of the GDPR (i.e. Articles 6-7, 12-22, 25, 29-30, 32-34). However, we did not consider Chapter V (Articles 44 - 50) since it focuses on legal administrative perspective for international border transfer rather than the software application level. Thus, we did not analyse them as software requirements in our study. For the ISO/IEC 29100 privacy framework, all of the contents were explored.

Thailand PDPA consists of 7 chapters, 96 articles and 7 rights of data subjects. We have analysed 16 articles in Chapter 2 (Personal data protection) and Chapter 3 (Rights of data subjects) which are the key chapters providing guidelines to govern personal data and privacy protection. Other chapters detailing the scope of use, definitions, assignment of personal data protection committee and supervisory authority, complaints and penalties are not included in the taxonomy development process as they are not related to software requirements. In the APEC privacy framework, there are four parts containing 72 points. We have analysed the APEC information privacy principles part as it is related to software requirements. Based on the scope defined above, we shortlisted 149 statements in GDPR, 63 in ISO/IEC 29100, 101 in Thailand PDPA and 74 in the APEC privacy framework to be explored. This initial set of statements was directly extracted from the list of itemised items and/or clauses in the selected parts of regulations and frameworks.

Next, we went through all the shortlisted statements. We analysed each statement using a set of pre-defined questions to identify relevant actions, involved/affected parties or objects and target results. A statement may cover more than one requirement.  The steps of privacy requirements identification process are explained below:

%Once we selected the articles and sections to work on, we went through all the sentences in those articles and sections. We analysed each sentence using a set of pre-defined questions to identify relevant actions, involved/affected parties or objects and target results. A sentence may cover more than one requirement. There are three key components for a requirement: 1) actions, 2) involved/affected parties or objects, and 3) target results. The detailed steps of privacy requirements identification process are explained below:

\begin{enumerate}[leftmargin=*, noitemsep]
	
	\item \textbf{Identifying actions:} We ask \textit{``Which action should be provided based on this statement?''} to identify the action associated with a requirement. Some examples of the action verbs used in the collected statements are: \textbf{ALLOW, COLLECT, ERASE, IMPLEMENT, INFORM, MAINTAIN, NOTIFY, OBTAIN, PRESENT, PROTECT, PROVIDE, REQUEST, SHOW, STORE, TRANSMIT and USE}.
	
	\item \textbf{Determining involved/affected parties or objects:} After an action is identified, we determine the object(s) of the action. The output from this step can be either involved/affected parties or objects that are directly identified or implied by the statements. The involved/affected parties can be any persons or stakeholders such as data subjects, data processors, data recipients, supervisory authorities or third parties. The question used to identify the involved/affected parties is ``Who is involved/affected from the statement?''. However, the objects are things that are created, processed or done by the actions specified in the statements (e.g. consent, preferences, personal data, functions and data repository). These objects are identified by asking ``What has to be created/done from the identified action?''. %If a statement refers to a data source or a data repository, it can be extracted as an object, while a data recipient can be extracted as the involved/affected party.
	
	\item \textbf{Considering the target result(s):} The target results refer to a goal that a statement aims to achieve. They can be identified by asking \textit{``What should be achieved based on the action of that statement?''} For example, Article 13(1)(c) in GDPR states ``..., the controller shall, at the time when personal data are obtained, provide the data subject with the purposes of the processing for which the personal data are intended as well as the legal basis for the processing;''. The goal in this statement is asking to provide the data subject with the purposes of processing. Hence, the purposes of the processing is a target result that the action verb \textit{PROVIDE} aims to achieve.
	
	\item \textbf{Structuring into a privacy requirement pattern:} The derived privacy requirement is coded in the format of action verb, followed by involved/affected parties or objects and target results.
	
\end{enumerate}

%In our study, the statements in the regulations, standards and frameworks are written in natural language, and they contain goals that shall be complied or satisfied. We derived those privacy goals and then constructed them as software privacy requirements. Thus, this method meets our objective in terms of the inputs that we have and outputs that we would like to achieve. There are three essential components in a statement in the privacy regulations, standards and frameworks: what to do (i.e. an action), to whom/what they apply (i.e. an affected/involved party or an object) and things they aim to achieve (i.e. an object complement). Our requirements identification process captured all of those three parts. Several studies also constructed privacy requirements by including these three components from the statements in the data protection regulations and standard \cite{Meis} and privacy policies \cite{Antn2004}.

The following examples illustrate these steps. A statement in the GDPR states ``... the controller shall, at the time when personal data are obtained, provide the data subject the identity and the contact details of the controller and, where applicable, of the controller's representative''. From this statement, we identify \textit{`PROVIDE'} as the action that the controller shall act. We then consider what should be provided by the controller, and that was \textit{`the identity and the contact details of the controller or the controller's representative'}. We determine the object responding to \textit{`to whom the identity and contact details of the controller or the controller's representative should be provided'}, and that is the data subjects. All three components formulate a privacy requirement as \textit{\textbf{PROVIDE} the data subjects with the identity and contact details of a controller/controller's representative (R22)}.

Another example is more complex than the previous one. In the GDPR, removing personal data is recommended in several ways such as: the data has been unlawfully processed; or the data subjects would like to erase their personal data themselves; or the system must erase personal data when the data subjects object to the processing; or the system must erase personal data when the data subjects withdraw consent; or the system must erase personal data when it is not necessary for the specified purpose(s); or the system must erase personal data when the purpose(s) for the processing has expired. We thus need to formulate different privacy requirements as they affect the ways that the functions would be provided to users in a system. In this example, the derived requirements are: \textit{\textbf{ERASE} the personal data when it has been unlawfully processed (R7)}, \textit{\textbf{ALLOW} the data subjects to erase his/her personal data (R44)}, \textit{\textbf{ERASE} the personal data when the data subjects object to the processing (R46)}, \textit{\textbf{ERASE} the personal data when a consent is withdrawn (R47)}, \textit{\textbf{ERASE} the personal data when it is no longer necessary for the specified purpose(s) (R52)} and \textit{\textbf{ERASE} the personal data when the purpose(s) for the processing has expired (R53)}. More examples of the privacy requirements derived in this step are included in the supplementary material.

%%Move to supplementary materials
%It is important to note that we have followed the GBRAM to extract and refine requirements from narrative statements. For some statements, they are straightforward since privacy requirements can be directly derived from them. However, we have restructured and refined some privacy requirements to emphasise the functionalities that should be provided by software systems. For example, Art. 7(1) in the GDPR states ``Where processing is based on consent, the controller shall be able to demonstrate that the data subject has consented to processing his or her personal data''. From this statement, we derive requirement \emph{R17 \textbf{SHOW} the relevant stakeholders the consent given by the data subjects to process their personal data}.
%
%The lawfulness of processing is one of the key principles for protecting privacy of data subjects. Based on the GDPR Art. 6, the processing of personal data will be lawful if the data subjects give consent \emph{or} the processing is required for other legal conditions, such as the processing is necessary for the performance of contract, compliance with a legal obligation and protecting vital interests. R35 in the taxonomy addresses the requirement of obtaining consent from the data subjects for the processing. For other legal conditions, the data controllers must inform the data subjects if they need to process personal data under those conditions. Requirements R38 and R39 in the taxonomy cover these scenarios.
%
%After lawfully obtaining personal data, the data controllers must provide the data subjects with mechanisms to execute their individual rights in the system. All the privacy requirements related to the rights of data subjects are covered in our taxonomy (i.e., the right to be informed (R12, R24, R26, R30, R31 and R50), the right of access (R1), the right to rectification (R45), the right to erasure (R44), the right to restriction of processing (R4), the right to data portability (R33), the right to object (R3), and the rights in relation to automated decision making and profiling (R21)).
%
%A number of requirements (e.g. R35, R39 and R60) can be triggered when the software is dealing with special category data/more sensitive data. In addition to consent, there are other conditions for a lawful processing of special categories of personal data (e.g. necessary for protecting vital interests and public interests). Requirement R35 covers the consent condition. R39 and R60 in our taxonomy address the remaining conditions specified in GDPR Art. 9. R39 requires the data controllers, if not obtaining the consent, must provide the data subjects the purpose(s) of the processing of special categories of personal data to the data subjects. The controllers also require to protect those personal data with appropriate measures (R60). The key privacy requirements related to international data transfers are also covered as follows: i) the data subjects must be informed about the transfer of their personal data to a third country or an international organisation (R9); ii) the personal data must be appropriately protected (R60) and iii) the transfer of personal data must comply with local requirements (R65).
%
%We note that an article, a section or a statement can lead to the identification of more than one requirements. For example, we derived 17 privacy requirements (i.e. R1-R4, R6, R9, R20-22, R27, R29, R34, R37, R39, R44-R45 and R55) from the Art. 13 in GDPR\footnote{See the file \emph{Privacy-requirements-references} in the replication package for more details \cite{reppkg-pridp}.}. The following example demonstrates two privacy requirements that were derived from a statement. A statement in Section 23-6 in Thailand PDPA states ``In collecting the Personal Data, the Data Controller shall inform the data subject, ... (5) information, address and the contact channel details of the Data Controller, where applicable, of the Data Controller's representative or data protection officer;'', we derived two requirements from this statement which are \textit{R20 PROVIDE the data subjects with the contact details of a data protection officer (DPO)} and \textit{R22 PROVIDE the data subjects with the identity and contact details of a controller/controller's representative}.
%
%We also note that we excluded the articles related to the DPOs in our study due to the following reasons. Firstly, the articles related to DPOs in GDPR mainly focus on their duties, tasks and responsibilities (i.e. Art. 37 - 39). The DPOs requirements are related to governance aspect rather than software requirements aspect. Secondly, the main role that directly determines the activities related to the processing of personal data is the data controllers. This makes the data controllers the key stakeholder in governing how personal data is processed and how the processing activities should be done in software development level. Finally, we have not found any DPOs requirements reported in issue reports in our study. This finding implies that the DPOs requirements were not reflected as software requirements in this context.

\textbf{Reliability assessment}: Three human coders, who are the co-authors of the paper, have independently followed the above process to identify privacy requirements from the GDPR and ISO/IEC 29100 privacy framework. All three coders had substantial software engineering background and at least 1 year of experience with data protection regulations and policies. The first author prepared the materials and detailed instructions\footnote{These are included in the replication package \cite{reppkg-pridp}.} for the process. The instructions were provided to all the coders before they started the identification process. A 1-hour training session was also held to explain the process of identifying privacy requirements, clarify ambiguities and define expected outputs. The coders also went through a few examples together to fine tune the understanding.

All the coders were provided with a form to record their results of each step. The form was pre-filled with 149 statements extracted from the GDPR and 63 statements from the ISO/IEC 29100 privacy framework. If a coder considers a statement as a privacy requirement, they need to identify the relevant components and structure it following the privacy requirement patterns above. Otherwise, they leave it blank. Initially, the three coders each respectively identified 100, 95 and 97 requirements from GDPR, and 36, 36 and 37 requirements from ISO/IEC 29100.

Since the requirements identified by the coders could be different, we used the Kappa statistic (also known as Kappa coefficient) to measure the IRR between the coders \cite{Viera2005}. The Kappa statistic ranges from -1 to 1, where 1 is perfect agreement and -1 is strong disagreement \cite{Viera2005}. There are several types of the Kappa statistics which suit different study settings \cite{Hallgren}. For this study, the Fleiss' Kappa was used as we had three coders coding the same datasets \cite{Fleiss1971}. The Kappa values were 0.8025 for GDPR and 0.7182 for ISO/IEC 29100, suggesting a substantial agreement level amongst all the coders \cite{Landis1977}. All the coders agreed that there were 43 and 20 statements from the GDPR and ISO/IEC 29100 respectively that are not privacy requirements. There were 20 GDPR statements (and 13 for ISO/IEC 29100) that the coders did not agree upon. Hence, a meeting session was held between the coders to discuss and resolve disagreements. 

The statements in Thailand PDPA and APEC privacy framework were identified by one of the coders using the same methodology performed with the GDPR and ISO/IEC 29100. We used only one coder because the provisions in Thailand PDPA and many principles in the APEC privacy framework share many commonalities with the GDPR and ISO/IEC 29100, respectively. We therefore decided that one coder would be sufficient. The coder, who was responsible for the Thailand PDPA and APEC privacy framework, was the main coordinator and also participated in the privacy requirements identification process for GDPR and ISO/IEC 29100. All the shortlisted statements in Thailand PDPA and APEC privacy framework were derived in this step. This brought the total number of privacy requirements obtained in this step to 249 (116 from GDPR, 33 from ISO/IEC 29100 and 55 from Thailand PDPA and 45 APEC privacy framework).

%The statements in Thailand PDPA and APEC privacy framework were identified by one of the coders using the same methodology performed with the GDPR and ISO/IEC 29100. All the shortlisted statements in Thailand PDPA and APEC privacy framework were derived in this step. This brought the total number of privacy requirements obtained in this step to 249 (116 from GDPR, 33 from ISO/IEC 29100 and 55 from Thailand PDPA and 45 APEC privacy framework).

%We used only one coder for the Thailand PDPA and APEC privacy framework because the statements are less complex than the GDPR and ISO/IEC 29100. The provisions in Thailand PDPA were largely written based on the GDPR \cite{Kateifides}, thus we did not require additional process or special needs as they share many commonalities. Similarly, many principles in the APEC privacy framework are similar to ISO/IEC 29100. We therefore decided that one coder would be sufficient. The coder, who was responsible for the Thailand PDPA and APEC privacy framework, was the main coordinator and also participated in the privacy requirements identification process for GDPR and ISO/IEC 29100. %We note that the privacy requirements identification process did not involve any legal experts. The results from this process was also not validated with any legal experts.

\newtext{\subsection{Privacy requirements refinement}} \label{subsec:req-refinement}

Requirements extracted either from the same or different documents can be similar, redundant or inconsistent. In this step, we identify those similar and duplicate requirements, and merge them into one single requirement. In case that the requirements are inconsistent, we perform further investigation and report for notice.

To identify and merge similar requirements, we first place those similar requirements into the same group. These requirements tend to achieve the same goal and have the same involved/affected parties or objects. We then determine the action and target result for the final merged requirement based on the following rules:
\begin{enumerate}[leftmargin=*, noitemsep]
	\item If the action verbs in the requirements are the same, we retain that action for the final merged requirement.
	\item If the actions are different, we consider the action verb based on the following:
	\begin{itemize}[leftmargin=*, noitemsep]
		\item Use \emph{ALLOW} if a requirement relates to data subject's ability to invoke his/her rights.
		\item Use \emph{PROVIDE} if a requirement aims to give information to stakeholders.
		\item Use \emph{OBTAIN} if a requirement aims to get a consent or permission from stakeholders.
		\item Use \emph{PRESENT} if a requirement aims to display options or choices to stakeholders. This action verb requires responses from the stakeholders (e.g. displaying toggles or radio buttons for users to select).
		\item Use \emph{SHOW} if a requirement aims to show information to stakeholders. This action verb does not require any responses from the stakeholders.
		\item Use \emph{NOTIFY} if a requirement aims to alert stakeholders.
		\item Use \emph{IMPLEMENT} if a requirement aims to build a mechanism to support an activity in a system.
		\item Use \emph{ERASE} if a requirement aims to erase data in software systems.
	\end{itemize}
	\item If the target results in the requirements are the same, we retain that target result for the final requirement.
	\item If the target results are different, we combine them together. In case they have redundant or synonymous words, we select one word from the words in the list.
\end{enumerate}	

We finally put together the action, involved/affected parties or objects and target results identified in the above steps to construct the final requirement. 

We have carefully investigated the terms and definitions used in GDPR and ISO/IEC 29100 and found that they mostly refer to similar or same things. For example, ISO/IEC 29100 defines PII as ``any information that (a) can be used to identify the PII principal to whom such information relates, or (b) is or might be directly linked to a principal''. Personal data in GDPR is defined as ``any information relating to an identified or identifiable natural person ('data subject')''. As can be seen, PII in ISO/IEC 29100 and personal data in GDPR in fact refer to the same thing, i.e. any information that can be used to identify or is linkable to a natural person. Similarly, a PII principal in ISO/IEC 29100 and a data subject in GDPR in fact refer to the same thing, i.e. a natural person who can be identified with identifiable information such as name.

Similarly, key terms GDPR and ISO/IEC 29100 are sometimes worded differently, however their definitions are similar. For example, \textit{`processing'} means ``any operation or set of operations which is performed on personal data or set of personal data, ...'' in GDPR and \textit{`processing of PII'} is defined as ``operation or set of operations performed upon personally identifiable information (PII)''. Another example is data controller in GDPR and PII controller in ISO/IEC 29100. Both terms refer to person that determines the purposes and means of the processing of personal data/PII. Other examples include personal data with PII, data subject with PII principal, data processor with PII processor, consent and third party. Thus, in the merging step, we use the terms from GDPR in representing our requirements in this taxonomy to avoid ambiguities. We have also found that privacy requirements derived from GDPR, ISO/IEC 29100, Thailand PDPA and APEC privacy framework are in fact at the same level of abstraction\footnote{see the supplementary material for more details.}.

%The additional example is a PII principal in ISO/IEC 29100 and a data subject in GDPR. A PII principal in ISO/IEC 29100 is defined as ``a natural person to whom the personally identifiable information relates''. A data subject in GDPR is defined as ``an identifiable natural person is one who can be identified, directly and indirectly, in particular by reference to an identifier such as a name, ...''. As can be seen, a PII principal in ISO/IEC 29100 and a data subject in GDPR in fact refer to the same thing, i.e. a natural person who can be identified with identifiable information such as name. 

%%We have found that software requirements derived from GDPR, ISO/IEC 29100, Thailand PDPA and APEC privacy framework are in fact at the same level of abstraction\footnote{see the file \emph{Privacy-requirements-abstraction} in the replication package for additional sample requirements \cite{reppkg-pridp}.}. \citeauthor{Meis} has also confirmed that GDPR and ISO/IEC 29100 are at the same level of abstraction \cite{Meis}. The following example demonstrates this case. We derived \textit{\textbf{PROVIDE} the data subject the recipients or categories of recipients of the personal data} from Art. 13(1)(e) in GDPR, \textit{\textbf{PROVIDE} the types of persons whom the PII can be transferred} from openness, transparency and notice principle in ISO/IEC 29100, \textit{\textbf{INFORM} the data subject the categories of Persons or entities to whom the collected Personal Data may be disclosed} from Section 23 in Thailand PDPA and \textit{\textbf{PROVIDE} the types of persons or organisations to whom personal information might be disclosed} from Point 21 in APEC framework. These four requirements demonstrate that they are at the same level of abstraction and aim to achieve the same goal. They can be merged in the requirements refinement process later.

%Table \ref{tab:GDPR-ISO-req-mapping} shows a few examples of these. This has also been confirmed in previous studies (e.g. \cite{Meis}).

\begin{comment}
	\begin{landscape}
		\begin{table}
			\centering
			\caption{Some sample requirements derived from GDPR, ISO/IEC 29100, Thailand PDPA and APEC privacy framework demonstrate that they are at the same level of abstraction.}
			\label{tab:GDPR-ISO-req-mapping}
			\begin{tabular}{p{4cm} p{1.25cm} p{4cm} p{1.25cm} p{4cm} p{1.25cm} p{4cm} p{1.25cm}}
				\toprule
				\textbf{Requirements derived from GDPR statements} & \textbf{GDPR reference} & \textbf{Requirements derived from ISO/IEC 29100} & \textbf{ISO/IEC 29100 reference} &
				\textbf{Requirements derived from Thailand PDPA} & \textbf{Thailand PDPA reference} &
				\textbf{Requirements derived from APEC framework} & \textbf{APEC framework reference} \\
				\midrule
				PROVIDE the existence of the right to withdraw consent & 13(2)(c)
				& ALLOW a PII principal to withdraw consent & 5.2
				& ALLOW the data subject to withdraw his or her consent & 19-5
				& None & None \\
				PROVIDE the data subject the recipients or categories of recipients of the personal data & 13(1)(e)
				& PROVIDE the types of persons whom the PII can be transferred & 5.8
				& INFORM the data subject the categories of Persons or entities to whom the collected Personal Data may be disclosed & 23-5
				& PROVIDE the types of persons or organisations to whom personal information might be disclosed & 21-4\\
				PROVIDE the data subject the purposes of the processing & 13(1)(c)
				& PROVIDE the PII principal the purpose of the processing of PII & 5.8
				& INFORM the data subject the purpose of the collection, use, or disclosure of the Personal Data & 19-3
				& PROVIDE the individuals with the purpose their information is to be used & (21-23)-1 \\
				PROVIDE the data subject the categories of personal data concerned & 15(1)(b)
				& PROVIDE the PII principal the specified PII required for the specified purpose & 5.8
				& INFORM the data subject the Personal Data to be collected & 23-4
				& PROVIDE the individuals with what personal information is collected & (21-23)-1 \\
				IMPLEMENT appropriate technical and organisational measures to protect personal data & 32(1)(a)
				& PROTECT PII with appropriate controls & 5.11
				& PROVIDE appropriate security measures for preventing the Personal Data & 37-2
				& IMPLEMENT organizational controls to prevent from the wrongful collection or misuse of personal information & 20-1, 28-1, 28-2\\ \bottomrule
			\end{tabular}
		\end{table}	
	\end{landscape}
\end{comment}


The following example demonstrates the requirements merging step. A statement in ISO/IEC 29100, ``... allow a PII principal to withdraw consent easily and free of charge ...'', derives a requirement \textit{\textbf{ALLOW} a PII principal to withdraw consent}. A statement in GDPR, ``... the controller shall ... provide the data subjects with ... the existence of the right to withdraw consent at any time ...'' gives a requirement \textit{\textbf{PROVIDE} the existence of the right to withdraw consent}. A statement in Thailand PDPA, ``The data subject may withdraw his or her consent at any time.'', gives a requirement \textit{\textbf{ALLOW} the data subject to withdraw his or her consent.} The goal of these three requirements is to let the PII principal/data subject withdraw consent, and the affected parties are PII principal and data subject. It is noted that we use the terms from GDPR for roles in our requirements (i.e. data subject, data controller, data processor and third parties). We therefore list them as similar requirements. The requirements have different actions (i.e. ALLOW and PROVIDE), we then use \emph{ALLOW} as the final action as these requirements are about data user's ability to withdraw consent. We acquire \textit{withdraw consent} as a common target result. Finally, we merge these three requirements into a single requirement: \textit{\textbf{ALLOW} the data subjects to withdraw consent (R6)}.

The duplicate requirements are the requirements that have the exact actions, involved/affected parties and target results. We represent these requirements as one requirement in the taxonomy. For example, we identify two exact requirements in the identification process, \emph{PROVIDE the data subject the categories of personal data concerned} in GDPR Art. 14(b) and 15(b). We retain one requirement (i.e. R42) in the taxonomy.

Requirements are inconsistent when they appear to contradict each other in performing the same actions. The following example demonstrates the consistency between the requirements in ISO/IEC 29100 and GDPR. We identify the requirements from ISO/IEC 29100 and GDPR as ``COLLECT only necessary PII for specific purposes'' and ``COLLECT the personal data as necessary for specific purposes'', respectively. Both requirements yield that the personal data must be collected as necessary for specific purposes. They are presented in both GDPR and ISO/IEC 29100. The requirements are therefore consistent, and merged as R41 COLLECT the personal data as necessary for specific purposes.

The following example is made up for the purpose of explanation to demonstrate the inconsistency between requirements. Assuming a statement states ``Any personal data can be freely collected without specifying a specific purpose for collection'', we have derived the requirement as ``COLLECT any personal data without a specific purpose''. This requirement would contradict with requirement R41 discussed above since the former does not require a specific purpose provided, while the latter does.

We merged in total 178 similar and duplicate requirements. We did not find any inconsistent requirements. For requirements traceability, we have provided a full list of the privacy requirements with their references to the GDPR articles, ISO/IEC 29100 principles, Thailand PDPA sections and APEC framework points in the replication package \cite{reppkg-pridp}. This step resulted in a final taxonomy of 71 privacy requirements in 7 goal categories which we will discuss in detail in the next subsection.

\newtext{\subsection{Privacy requirements classification}} \label{subsec:req-classification}

In this step, we aim to group the privacy requirements into categories based on their goals. \newtext{We adopted the empirical-to-conceptual approach to develop our taxonomy \cite{Nickerson}. We identified the privacy requirements that had common characteristics, grouped those privacy requirements together, and finally formed the categories. As we have classified the privacy requirements from their smallest unit of analysis (i.e. each privacy requirement), thus we called this approach as the bottom-up approach.} This approach ensures that the generated categories cover and address all the requirements. The approach also allows the categories in the taxonomy to be updated when there are new privacy requirements identified in the future. For example, newly identified privacy requirements can be added to existing categories or form new categories.

The bottom-up approach consists of two steps. We first considered the privacy requirements based on their actions, objects and target results. For example, the privacy requirements with the action verb \emph{ALLOW}, the object \emph{data subjects} and the target results that are related to individual rights (e.g. access, rectify and erase) were grouped together (i.e. user participation). Similarly, the privacy requirements with the action verb \emph{PROVIDE}, the object \emph{data subjects} and the target results that are related to information for stakeholders were gathered into the same group. We kept applying this strategy to group the rest of privacy requirements. We then ended up with fifteen categories in the first step.

Next, we grouped the privacy requirements that have at least two common components either the actions, objects or target results. For example, we gathered the privacy requirements that have the same action verb \emph{PROVIDE} and the same target results that are related to information for stakeholders, but there are two different objects - data subjects and other parties that are not data subjects. We then created subcategories for each object (i.e. notice - data subjects and notice - relevant parties) \newtext{(see Subcategories 2.1 and 2.2 in Table \ref{tab:sample-requirements})}. All of these requirements were grouped under the notice category. We repeated this step with the rest of privacy requirements. Finally, the taxonomy consists of seven categories (some of which have sub-categories): user participation, notice, user desirability, data processing, breach, complaint/request and security. 

After we had categorised all the privacy requirements, we noticed that some of them address more than one category. For example, requirement \emph{R6 ALLOW the data subjects to withdraw consent} addresses both user participation and also consent (which is under user desirability) categories. Thus, we again went through all the privacy requirements, and considered if the privacy requirements address other relevant categories other than their existing one. We added those privacy requirements into other relevant categories. Thus, some requirements can belong to multiple categories. The descriptions of privacy goal categories and the examples of privacy requirements in each goal category are explained in detail in Section \ref{sec:taxonomy}. \\

\newtext{We note that the mutual exclusivity may be desirable for some taxonomies, however it is not compulsory as long as the taxonomy is useful \cite{Nickerson}. We demonstrated our taxonomy is useful based on the following attributes:}

\begin{itemize}[leftmargin=*,noitemsep]
	\item \newtext{Concise: our taxonomy consists of limited categories to cover all privacy requirements.}
	\item \newtext{Robust: the categories in the taxonomy clearly and adequately differentiate the privacy requirements into specific and relevant groups.}
	\item \newtext{Comprehensive: our taxonomy can classify the privacy requirements within the concerned principles and properties of the data protection and privacy regulations and frameworks.}
	\item \newtext{Extensible: the approach we adopted in the study allows the categories in the taxonomy to be updated when there are new privacy requirements identified in the future.}
	\item \newtext{Explanatory: our categories have their own characteristics and are able to provide specific explanation to describe the privacy requirements under them.}
\end{itemize}

\begin{comment}
	
	The descriptions of privacy goal categories below provide an overview and briefly illustrate the privacy requirements concerned in those categories. We note that the descriptions do not explain the full list of privacy requirements in each category.
	
	\begin{enumerate}[leftmargin=*, noitemsep]
		
		\item User participation \\
		This privacy goal consists of a set of requirements for the controllers to provide the data subjects with the functionalities to invoke their individual rights relating to their personal data. The data subjects must be able to access, review, rectify, erase and verify the validity and completeness of their personal data. The controllers shall provide the data subjects their personal data when they request to obtain and reuse their personal data for their own purposes for other services. The controllers shall allow the data subjects to object to and restrict the processing of their personal data. The data subjects must be able to withdraw consent or lodge a complaint to a supervisory authority.
		
		\item Notice \\
		This privacy goal consists of two sub-categories: data subjects and relevant parties. This category has a set of requirements for the data subjects to be informed and/or notified of relevant information and individual rights related to the processing of personal data. The information includes privacy policies, procedures, practices and logic of the processing of personal data. The data subjects must be informed of the purposes of collection and processing of their personal data. The controllers must provide the contact details of responsible persons who control the processing. The data subjects must be notified when they are likely to be in risk from personal data exposures. In addition, the controllers must communicate any relevant information relating to personal data processing to intended stakeholders.
		
		\item User desirability \\
		This privacy goal consists of three sub-categories: consent, choice and preferences. This category asserts that the controllers must show a consent form to and obtain consent from the data subjects. The controllers must provide the data subjects with an option to provide their data, allow the processing or subject to a decision based on automated processing. The processing should be implemented based on user preferences expressed in their consent.
		
		\item Data processing \\
		The privacy goal addresses the processing of personal data handling from the controllers' side. The processes, which are the sub-categories in this category, include collection, use, storage, erasure, transfer and record. The controllers must handle personal data as necessary for specific purposes. The processors who process personal data must also follow the instructions from the controllers. The personal data processing activities must be recorded.
		
		\item Breach \\
		This goal category ensures that the controllers must be prepared to handle personal data breach. The relevant information about the data breach must be recorded and communicated to data subjects and relevant stakeholders.
		
		\item Complaint/Request \\
		This goal category addresses complaint and request management. User complaints and requests about their individual rights and the processing of their personal data must be processed. The actions regarding the complaints and requests must be informed. The controllers must request for relevant information to confirm the identity of data subjects when the request has been made.
		
		\item Security \\
		This privacy goal ensures that personal data and its processing are safeguarded with confidentiality, integrity and availability. The personal data must be protected with appropriate controls and security mechanisms. The security mechanisms must comply with security and data protection standards. The personal data must be accessed and used by the authorised stakeholders. The controllers must ensure the correctness and completeness of personal data. 
	\end{enumerate}
	
\end{comment}

%When software engineers map an issue report to relevant privacy requirements, they can consider the features/concerns that the issue report mentions based on categories, then explore relevant requirements under the selected categories and sub-categories. The software engineers will know what has been missed in this issue, so they can resolve the issue accordingly. For example, an issue report reports on not allowing users to modify their personal data, the software engineer can directly observe the requirements in the user participation category as it relates to the data subject's right. The software engineer then identifies the relevant requirement of this issue report, which in this case is requirement R45. Similarly, if the issue mentions about consent, the engineer can access the requirements related to consent in the consent sub-category, select the relevant one(s) and resolve the issue based on the identified privacy requirements.

Our methodology is also able to address the scenario where two requirements aiming to achieve the same goal have different actors. We can further refine relevant requirements into a parent requirement and child requirements with specific logical operators (i.e. AND and OR). \newtext{Since we did not find any case where two requirements aiming to achieve the same goal have different actors in the privacy requirements extracted from GDPR, ISO/IEC 29100 Thailand PDPA or APEC framework, we used a sample of privacy requirement extracted from other regulations to illustrate the privacy requirement related to obtaining consent instead.} For example, assume that the controller is required to obtain consent for the processing of personal data. The data controller is responsible for this task in GDPR, however this could be managed by 3rd party authority in other regulations (e.g. CCPA). We can then refine these requirements into a parent requirement as \textit{obtain consent for the processing of personal data}. The child requirements could be expressed as \textit{the controllers shall obtain consent for the processing of personal data} and \textit{the 3rd party authority shall obtain consent for the processing of personal data} (see Figure \ref{fig:different-actors}). The logical operator in this scenario is \textit{OR} as software development teams can choose between one of the child requirements to implement to satisfy the parent requirement.

%Revised: This proposed methodology is able to address the scenario where two requirements aiming to achieve the same goal have different actors. We can further refine relevant requirements into a parent requirement and child requirements with specific logical operators (i.e. AND and OR). Since we did not find any case where two requirements aiming to achieve the same goal have different actors in the privacy requirements extracted from GDPR, ISO/IEC 29100 Thailand PDPA or APEC, we used a sample of privacy requirement extracted from other regulations to illustrate the privacy requirement related to obtaining consent instead. For example, assume that the controller is required to obtain consent for the processing of personal data. The data controller is responsible for this task in GDPR, however this could be managed by 3rd party authority in other regulations (e.g. California Consumer Privacy Act (CCPA) ). We can then refine these requirements into a parent requirement as \textit{obtain consent for the processing of personal data}. The child requirements could be expressed as \textit{the controllers shall obtain consent for the processing of personal data} and \textit{the 3rd party authority shall obtain consent for the processing of personal data} (see Figure \ref{fig:different actors}). The logical operator in this scenario is \textit{OR} as software development teams can choose between one of the child requirements to implement to satisfy the parent requirement.

If a regulation states that it requires both the controllers and the 3rd party authority to obtain consent for the processing of personal data, the logical operator in this scenario is \textit{AND}, and the software development teams must implement both child requirements in their system. As a data controller is the main actor in the regulations and frameworks we analysed in this study, the above scenario did not occur in our study. Thus, we did not refer to a specific requirement in our taxonomy.

\begin{figure}[h]
	\centering
	\includegraphics[width=1\linewidth]{"Figures/different-actors"}
	\caption{\newtext{An example demonstrates a scenario where two requirements aiming to achieve the same goal having different actors}}
	\label{fig:different-actors}
\end{figure}

\vspace{-4mm}

\begin{comment}
	\textbf{Reliability assessment}: The same group of coders took part in this step. The instructions for privacy requirements classification were provided to all the coders before they started the classification process. They were also provided a form containing a list of privacy requirements as agreed upon from the previous step and choices of privacy goal categories to be selected. All the coders independently classified each requirement into a privacy goal category.
	
	After all the coders finished the classification, the reliability assessment was conducted to evaluate the inter-rater agreement amongst the coders. We also used Fleiss' Kappa as a measure. The Kappa value was 0.7132, suggesting a substantial agreement level \cite{Landis1977}. There were 39 requirements that the coders did not classify into the same category. A meeting session was hold between the coders to discuss and resolve disagreements. This step resulted in a set of 149 privacy requirements classified into seven privacy goal categories.
\end{comment}