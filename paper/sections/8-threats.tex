\section{Threats to validity} \label{sec:threats}

%\textbf{Internal validity:} Threats to internal validity relate to coders' interpretation.

Our study involved subjective judgements. We have applied several strategies to mitigate this threat such as using multiple coders (who are the authors of the paper), applying IRR assessments, organising training sessions and disagreement resolution meetings. A legal expert could have extended the view of the human coders in interpreting legal regulations. However, we note that all the coders had attended a training course on privacy regulations (including GDPR). This has enhanced our interpretation of privacy regulations from a legal perspective, thus minimised this risk. In addition, our study was built upon well-founded processes and theories in previous work such as Grounded Theory \cite{Glaser2017} and GBRAM \cite{Antn2004}. We acknowledge that other contemporary methods could be used to extract requirements from legal texts (e.g. \cite{Breaux2006}, \cite{Ghanavati2009}). We also used relevant statistical measures and techniques to ensure that our findings are statistically significant. \\
%The coding process with IRR assessments was performed with the GDPR and ISO/IEC 29100 only.  \\
%Our taxonomy of requirements is derived from the GDPR and ISO/IEC 29100 -- two major and widely-adopted privacy and data protection regulations and framework to date; and Thailand PDPA and APEC privacy framework -- two region-specific representative regulations and privacy frameworks. 
\indent We are aware that there are other privacy protection laws and regulations applied around the world. However, most of them share many commonalities with the GDPR and ISO/IEC 29100 as confirmed by Thailand PDPA and APEC privacy framework in this study. 
%In fact, previous studies (e.g. \cite{Linden2020}, \cite{Tsohou2020}) have shown that GDPR is one of the most comprehensive data protection regulations. 
In fact, GDPR is one of the most comprehensive data protection regulations \cite{Linden2020}, \cite{Tsohou2020}. It was also used as a benchmark for other countries to develop data protection regulations such as Japan, South Korea and Thailand \cite{Torre}, \cite{Laboris2019}. Hence, although the principles or rights in other laws and regulations are slightly different due to variations in each country/city, they share many commonalities with GDPR (e.g. see the comparison between GDPR and CCPA in \cite{DataPrivacyManager}). ISO/IEC 29100 has also been used to develop organisational and technical privacy controls in many information and communication systems \cite{PECB2015}. Therefore, we found that GDPR and ISO/IEC 29100 together are the most comprehensive, thus our taxonomy can generalise to other privacy regulations and standards. We acknowledge that future research could involve investigating country specific privacy regulations, and extending our taxonomy accordingly.  \\
%Many other data protection laws have been developed as inspired by GDPR (e.g. CCPA in California) \cite{GDPR.EU2020}. It was also used as a benchmark for other countries to develop data protection regulations such as Japan, South Korea and Thailand \cite{Torre}, \cite{Laboris2019}. Hence, although the principles or rights in other laws and regulations are slightly different due to variations in each country/city, they share many commonalities with GDPR (e.g. see the comparison between GDPR and CCPA in \cite{DataPrivacyManager}). ISO/IEC 29100 has also been used to develop organisational and technical privacy controls in many information and communication systems \cite{PECB2015}. Therefore, we found that GDPR and  ISO/IEC 29100 together are the most comprehensive, thus our taxonomy can generalise to other privacy regulations and standards. We acknowledge that future research could involve investigating country specific privacy regulations, and extending our taxonomy accordingly. We also note that developing a taxonomy of privacy requirements for GDPR, ISO/IEC 29100, Thailand PDPA and APEC privacy framework already required substantial efforts and detailed processes. \\
\indent It is noted that the taxonomy does not address all the levels of abstractions of privacy requirements. The privacy requirements can be refined into a subset of other requirements. For example, the notion of consent as a privacy requirement can be refined into multiple requirements such as consent methods and properties of consent management platform. This depends on business requirements, organisational processes and software development teams which limit the applicability of our taxonomy. Also, the future amendments to the regulations and frameworks may require updates of our taxonomy. \\
\indent \newtext{In a modern software development, requirements collected during the requirement elicitation phase can be specified and recorded in issue-tracking systems (such as JIRA). However, we acknowledge that issue reports may not capture all requirements of a software application, e.g. requirements associated with the software's core architecture. Our methodology can be extended to map those requirements into the privacy taxonomy. Future work would involve combining our approach with other sources of software requirements to provide a more complete view of how a software application addresses the privacy requirements in the taxonomy.}

Finally, we performed the mining and classification of issue reports for Chrome and Moodle. These are two large and widely-used software systems that have strong emphasis on privacy concerns. However, we acknowledge that our datasets may not be representative of other software applications. Further investigation is required to explore other projects in different domains (e.g. e-health software systems and mobile applications). However, we note that building this dataset on two projects alone required substantial effort and highly thorough processes (278 person-hours).

